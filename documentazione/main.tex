%% Copyright 2007-2020 Elsevier Ltd
%% Template modificato per layout professionale a singola colonna
\documentclass[12pt]{report}

%% Pacchetti essenziali
\usepackage{amssymb}
\usepackage{lipsum}
\usepackage[utf8]{inputenc}
\usepackage[italian]{babel}
\usepackage[T1]{fontenc}
\usepackage{graphicx}
\usepackage{url}
\usepackage{algorithm}
\usepackage{algorithmic}
\usepackage{amsmath}
\usepackage{tikz}
\usepackage{multirow}
\usetikzlibrary{shapes.geometric, arrows, positioning, fit, backgrounds, calc}
\usepackage{hyperref}
\usepackage{fancyhdr}
\usepackage{geometry}
\usepackage{tcolorbox}
\usepackage{booktabs}
\usepackage{caption}
\usepackage{float}

% Per tabelle colorate (cellcolor)
\usepackage{colortbl}

%% Impostazioni geometria pagina
\geometry{a4paper, margin=2.5cm}

%% Impostazioni header e footer
\pagestyle{fancy}
\fancyhf{}
\rhead{Progetto HeRoN}
\lhead{Adaptive Decision Making NPC}
\cfoot{\thepage}
\setlength{\headheight}{14.5pt}

%% Comandi personalizzati
\newcommand{\kms}{km\,s$^{-1}$}

% Ridefinizione ambiente abstract per "Abstract" invece di "Sommario"
\addto\captionsitalian{%
  \renewcommand{\abstractname}{Abstract}%
}

% ...existing code...

\begin{document}

%% ----------------------------------------------------------------------------
%% FRONTESPIZIO PERSONALIZZATO
%% ----------------------------------------------------------------------------
\begin{titlepage}
    \centering
    \vspace*{2cm}
    
    % Logo università (decommenta se disponibile)
    % \includegraphics[width=0.25\textwidth]{logo_unisa.png}\\[1.5cm]
    
    {\Large\textsc{Dipartimento di Informatica}}\\[0.3cm]
    {\large\textsc{Università degli Studi di Salerno}}\\[0.5cm]
    {\large Corso di Intelligenza Artificiale}\\[3cm]
    
    % Titolo principale
    {\Huge\bfseries Adaptive Decision Making\\[0.4cm]
    NPC in Crafter}\\[1.5cm]
    
    \rule{\linewidth}{0.3mm}\\[0.5cm]
    {\Large\textbf{Architettura HeRoN per Reinforcement Learning}}\\[0.5cm]
    \rule{\linewidth}{0.3mm}\\[3cm]
    
    % Autori
    \begin{minipage}{0.8\textwidth}
        \centering
        \textbf{\large Realizzato da:}\\[1cm]
        
        \begin{tabular}{ll}
            \textsc{Danilo Gisolfi} & Matricola: 0522502001 \\[0.5cm]
            \textsc{Vincenzo Maiellaro} & Matricola: 0522502055 \\
        \end{tabular}
    \end{minipage}
    
    \vfill
    
    % Data
    {\large Anno Accademico 2025/2026}
    
\end{titlepage}

%% ----------------------------------------------------------------------------
%% ABSTRACT
%% ----------------------------------------------------------------------------
\begin{abstract}
Questo lavoro presenta l'adattamento e la validazione dell'architettura HeRoN (Helper-Reviewer-NPC) all'environment Crafter, un survival game open-world che rappresenta un benchmark significativo per il Reinforcement Learning. L'architettura integra tre componenti principali: un agente Deep Q-Network (DQN) con Double DQN e Prioritized Experience Replay, un Helper basato su Large Language Model (Qwen3-4B-2507) che genera sequenze strategiche di 3-5 azioni, e un Reviewer fine-tuned (FLAN-T5-base) che fornisce feedback correttivi per migliorare i suggerimenti dell'Helper.

L'implementazione affronta sfide tecniche complesse: la sparsità del reward nativo di Crafter viene mitigata attraverso un sistema di reward shaping multi-componente; la gestione delle situazioni critiche è garantita da meccanismi di re-planning che interrompono le sequenze pre-pianificate in caso di emergenza; le allucinazioni dell'LLM sono corrette mediante un sistema di validazione e fuzzy matching che porta la percentuale di azioni valide dal 87\% al 98\%.

Lo spazio di stati è rappresentato da un vettore a 43 dimensioni comprendente inventario (16 item), posizione, statistiche vitali e achievement sbloccati (22 obiettivi). Il training integrato utilizza un meccanismo di threshold decay che riduce progressivamente l'intervento LLM da 100\% a 0\% nei primi 100 episodi, permettendo all'agente DQN di acquisire gradualmente autonomia.

I risultati sperimentali includono metriche di apprendimento dettagliate, analisi approfondite delle policy miste RL+LLM e strumenti avanzati per la visualizzazione e la valutazione degli apprendimenti conseguiti dall'architettura proposta.

\vspace{0.5cm}
\end{abstract}

%% ----------------------------------------------------------------------------
%% INDICE
%% ----------------------------------------------------------------------------
\newpage
\tableofcontents
\newpage

% Lista delle figure
\listoffigures
\newpage
% Lista delle tabelle
\listoftables
\newpage

%% ----------------------------------------------------------------------------
%% CONTENUTO (capitoli da includere)
%% ----------------------------------------------------------------------------
\chapter{Introduzione}

\section{Contesto}

Il presente lavoro rientra nel campo del Reinforcement Learning applicato ai videogiochi, un'area di ricerca in rapida crescita che mira a creare agenti intelligenti capaci di imparare strategie ottimali interagendo con ambienti di gioco.

I videogiochi moderni, soprattutto quelli open-world e di sopravvivenza, presentano sfide complesse che richiedono agli agenti di prendere decisioni strategiche a lungo termine, gestire risorse limitate e adattarsi a situazioni dinamiche. Questi ambienti sono perfetti per testare e validare nuove idee di intelligenza artificiale.

\section{Motivazione e Obiettivi}

L'architettura HeRoN (Helper-Reviewer-NPC) è un approccio innovativo che combina il Reinforcement Learning tradizionale con il ragionamento dei Large Language Model (LLM). Questa architettura è stata inizialmente validata in environment di tipo JRPG (Japanese Role-Playing Game) a turni, dimostrando efficacia nel migliorare le prestazioni degli agenti RL. 

La sfida principale consiste nell'estensione di HeRoN a un contesto molto diverso: il gioco Crafter, un open-world di sopravvivenza che richiede pianificazione a lungo termine, gestione delle risorse e adattamento dinamico. 

\subsection{Obiettivi Primari}

Gli obiettivi principali del lavoro sono:

\begin{itemize}
    \item \textbf{Adattamento architetturale}: Estendere HeRoN dall'environment JRPG a turni a Crafter, un survival game open-world in tempo continuo
    \item \textbf{Fine-tuning del Reviewer}: Adattare il componente Reviewer ai task specifici di Crafter, generando un dataset appropriato e addestrando il modello per feedback efficaci nel survival game
    \item \textbf{Generazione di sequenze}: Modificare l'Helper per generare sequenze di 3-5 azioni coerenti anziché singole decisioni
    \item \textbf{Implementazione DQN}: Sviluppare un agente RL basato su Deep Q-Network ottimizzato per le 17 azioni disponibili e spazio di stati a 43 dimensioni
    \item \textbf{Valutazione comparativa}: Valutare HeRoN quantitativamente rispetto a baseline tradizionali, misurando achievement sbloccati nei 22 obiettivi disponibili
\end{itemize}

\subsection{Obiettivi Secondari}

\begin{itemize}
    \item Determinare il numero ottimale di azioni per sequenza dell'Helper
    \item Analizzare l'impatto del reward shaping sulle prestazioni dell'agente
    \item Implementare meccanismi di re-planning per situazioni critiche (salute bassa, achievement sbloccati)
\end{itemize}

%\subsection{Struttura del Documento}
%
%Il presente documento è organizzato come segue:
%
%\begin{itemize}
%    \item \textbf{Sezione 2 - Architettura HeRoN}: Descrizione dettagliata dell'architettura HeRoN, dei suoi tre componenti (NPC, Helper, Reviewer) e delle loro interazioni.
%    
%    \item \textbf{Sezione 3 - Environment Crafter}: Presentazione dell'environment Crafter, delle sue caratteristiche, dello spazio di stati e azioni, e dei 22 achievement disponibili.
%    
%    \item \textbf{Sezione 4 - Metodologia}: Descrizione della metodologia di implementazione, dalla preparazione dell'environment al training dell'architettura completa.
%    
%    \item \textbf{Sezione 5 - Risultati}: Presentazione e analisi dei risultati sperimentali, con confronti quantitativi tra HeRoN e baseline.
%    
%    \item \textbf{Sezione 6 - Conclusioni}: Sintesi dei risultati ottenuti, discussione delle sfide affrontate e prospettive future.
%\end{itemize}

\chapter{Architettura HeRoN}

\section{Panoramica dell'Architettura}

HeRoN (Helper-Reviewer-NPC) è un'architettura multi-agente che combina Reinforcement Learning e Large Language Model per migliorare il processo decisionale di agenti intelligenti in ambienti interattivi. L'idea di fondo consiste nell'unire la capacità del Reinforcement Learning di ottimizzare strategie attraverso prove ed errori, il ragionamento semantico e la conoscenza generale dei Large Language Model, e un meccanismo di feedback iterativo per migliorare i suggerimenti.

\subsection{Diagramma Architettura DQN Baseline}

Prima di descrivere l'architettura completa HeRoN, viene presentata l'architettura baseline DQN utilizzata come riferimento per il confronto con l'integrazione LLM.


\textbf{Flusso Operativo DQN}:
\begin{enumerate}
    \item \textbf{Percezione}: Ambiente → Estrazione dello stato (vettore 43-dim)
    \item \textbf{Decisione}: Rete DQN → Q-values → selezione $\epsilon$-greedy
    \item \textbf{Azione}: Esegui azione $a_t$, osserva $r_t, s_{t+1}$
    \item \textbf{Memorizzazione}: Salva $(s_t, a_t, r_{shaped}, s_{t+1}, done)$ in Prioritized Replay
    \item \textbf{Apprendimento}: Campiona batch → calcola TD-loss → aggiorna pesi DQN
    \item \textbf{Stabilizzazione}: Ogni 100 passi, copia pesi DQN → Rete Target
\end{enumerate}


L'architettura DQN baseline apprende esclusivamente tramite interazione diretta con l'ambiente, senza supporto esterno.

\subsection{Diagramma Architettura HeRoN Completa}


L'architettura HeRoN rappresenta un'estensione del DQN baseline, con l'aggiunta di due componenti LLM per la guida strategica e un meccanismo di \textbf{threshold decay} che bilancia l'intervento tra LLM e RL.

\textbf{Meccanismo Threshold Decay}:

Il threshold $\theta$ controlla quando consultare il LLM anziché usare DQN autonomo. Nel progetto sono state implementate tre strategie di attivazione LLM:

\begin{enumerate}
    \item \textbf{HeRoN Initial}: LLM attivo solo nei primi 100 step di ogni episodio (finestra temporale fissa)
    \item \textbf{HeRoN Random}: LLM con probabilità casuale del 50\% ad ogni step (attivazione stocastica)
    \item \textbf{HeRoN Final}: Threshold decay adattivo $\theta(t) = \max(0, 1.0 - k \times t)$ con $k=0.01$ (probabilità LLM crescente 0\%→100\% durante ogni episodio)
\end{enumerate}

Tutte le configurazioni includono un \textbf{cutoff a episodio 100}: dopo i primi 100 episodi, il DQN diventa completamente autonomo.

\textbf{Flusso Decisionale Integrato}:
\begin{enumerate}
    \item Ambiente genera stato → Estrazione dello stato (43-dim)
    \item Threshold check: random(0.73) $>$ threshold(0.65) → \textbf{LLM Path}
    \item Helper riceve descrizione dello stato → genera \texttt{[move\_right], [do],} \\
    \texttt{[move\_left], [do], [noop]}
    \item Reviewer analizza → feedback: \textit{"Health low, prioritize eating"}
    \item Helper re-query con feedback → sequenza raffinata: \texttt{[sleep],} \\
    \texttt{[move\_right], [do], [move\_left], [noop]}
    \item Action Executor esegue \texttt{[sleep]} → $(s_1, r_1, info_1)$
    \item Salva $(s_0, sleep, r_1, s_1, done)$ in Prioritized Replay
    \item Monitor: achievement unlocked? → \textbf{SÌ} → interrompi sequenza, nuova query Helper
    \item DQN training: sample batch → compute loss → aggiorna pesi
\end{enumerate}


L'architettura HeRoN integra i vantaggi del Reinforcement Learning (apprendimento da esperienza) e dei Large Language Model (conoscenza a priori e ragionamento strategico), con una transizione graduale verso l'autonomia dell'agente.

L'architettura HeRoN è composta da tre componenti principali che interagiscono in modo coordinato:

\subsection{NPC (Non-Player Character)}

L'NPC è l'agente che gioca in Crafter usando Reinforcement Learning. In questo progetto, l'NPC implementa l'algoritmo \textbf{Deep Q-Network (DQN)} con le seguenti caratteristiche:

\begin{itemize}
    \item \textbf{Architettura}: Rete neurale feedforward a 3 hidden layers (43-128-128-64-17) per mappare stati a Q-values
    \item \textbf{Double DQN}: Due reti distinte (policy network e target network) per stabilizzare l'apprendimento
    \item \textbf{Prioritized Experience Replay}: Campionamento intelligente delle esperienze passate basato su TD-error
    \item \textbf{Funzionamento}: L'NPC osserva lo stato (43-dim), seleziona un'azione tramite strategia $\epsilon$-greedy, esegue l'azione, riceve reward e aggiorna i pesi della rete neurale
\end{itemize}


\subsection{Helper}

Il componente Helper è un Large Language Model utilizzato in modalità zero-shot che fornisce suggerimenti strategici all'NPC. Nel progetto HeRoN per Crafter, l'Helper si implementa utilizzando un LLM locale (Qwen3-4B-2507) attraverso LM Studio con le seguenti caratteristiche:

\begin{itemize}
    \item \textbf{Generazione di sequenze di azioni}: Diversamente dall'implementazione originale che suggeriva singole azioni, l'Helper in questo progetto genera sequenze di 3-5 azioni coerenti da eseguire una dopo l'altra.
    
    \item \textbf{Contestualizzazione}: L'Helper riceve informazioni dettagliate circa lo stato corrente del gioco:
    \begin{itemize}
        \item Inventario del giocatore (16 item)
        \item Posizione corrente
        \item Statistiche vitali (salute, cibo, acqua)
        \item Achievement sbloccati (22 possibili)
    \end{itemize}
    
    \item \textbf{Prompt Engineering}: Il prompt si presenta come specificatamente progettato per Crafter e include:
    \begin{itemize}
        \item Descrizione del contesto di gioco
        \item Stato corrente dell'agente
        \item Lista delle azioni disponibili
        \item Richiesta di generare una sequenza strategica
    \end{itemize}
    L'Helper risponde con una sequenza di azioni nel formato:
    \begin{verbatim}
    [azione_1], [azione_2], [azione_3], [azione_4], [azione_5]
    \end{verbatim}
    Ad esempio:
    \begin{verbatim}
    [move_right], [do], [move_left], [do], [noop]
    \end{verbatim}
\end{itemize}

\subsection{Reviewer}

Il componente Reviewer è un LLM fine-tuned (basato su T5) che valuta i suggerimenti forniti dall'Helper e genera feedback per migliorarli. Come descritto in dettaglio nel Capitolo~4, il Reviewer si addestra specificamente per il contesto di Crafter su un dataset composto da 150 episodi di gioco, contenente circa 15.000 esempi di coppie (suggerimento, feedback).

Il Reviewer analizza:
\begin{itemize}
    \item Coerenza della sequenza di azioni suggerite
    \item Appropriatezza rispetto allo stato corrente
    \item Potenziali rischi o inefficienze
    \item Priorità strategiche (es. sopravvivenza vs. progressione)
    \item Fornisce feedback strutturato che si utilizza per ri-interrogare l'Helper con più informazioni.
\end{itemize}

\subsection{Gestione del Contesto}

Durante ogni episodio, l'Helper mantiene uno stato conversazionale persistente (\texttt{message\_history}) che accumula descrizioni dello stato di gioco, sequenze di azioni generate, feedback del Reviewer e contesto di gioco (posizione, inventario, achievement). Questa memoria conversazionale consente all'Helper di mantenere coerenza e contestualizzazione lungo l'episodio, influenzando direttamente la qualità dei suggerimenti generati.

La cronologia consente al LLM di mantenere coerenza logica tra azioni successive e di comprendere l'evoluzione dello stato di gioco all'interno dell'episodio.

\subsection{Gestione Intelligente del Contesto (Token-Aware)}

L'Helper implementa un sistema di gestione intelligente del contesto per prevenire l'overflow della finestra di contesto del modello LLM (Qwen3-4B-2507 ha 8192 token di limite):

\begin{enumerate}
    \item \textbf{Monitoraggio token}: L'Helper conta il numero di token nella cronologia utilizzando il tokenizer Qwen2.5 (compatibile con Qwen3)
    
    \item \textbf{Soglia di sicurezza}: Quando il contesto raggiunge 6500 token (80\% del limite), viene attivato un reset intelligente per evitare crash e risposte vuote
    
    \item \textbf{Reset con riassunto episodio}: Invece di scartare tutto il contesto, l'Helper genera un riassunto che include:
    \begin{itemize}
        \item Numero di step eseguiti nell'episodio
        \item Reward totale accumulato
        \item Lista degli achievement sbloccati
        \item Feedback recente del Reviewer (se disponibile)
        \item Descrizione dello stato di gioco corrente
    \end{itemize}
    Questo riassunto diventa il nuovo inizio della cronologia, preservando informazioni strategiche critiche.
\end{enumerate}

\textbf{Vantaggi del reset intelligente}:
\begin{itemize}
    \item \textbf{Continuità strategica}: Il modello comprende il progresso episodico complessivo
    \item \textbf{Efficienza token}: Riduce i token inutili mantenendo informazioni essenziali
    \item \textbf{Riduzione allucinazioni}: Contesto pulito riduce risposte errate o non coerenti
    \item \textbf{Maggiore lunghezza episodio}: Consente episodi più lunghi senza crash
\end{itemize}

\subsection{Meccanismi di Re-planning e Aggiornamento Contesto}

Durante l'esecuzione di una sequenza di azioni, l'Helper monitora determinati eventi per aggiornare intelligentemente il contesto:

\textbf{Trigger di Re-query (Interruzione Sequenza)}:
\begin{itemize}
    \item \textbf{Achievement sbloccato}: Quando il giocatore sblocca un nuovo achievement, l'Helper riceve una nuova query con il contesto aggiornato che include il nuovo achievement nel set di quelli sbloccati
    
    \item \textbf{Salute critica} ($health \leq 5$): Se la salute scende sotto soglia critica, la sequenza viene interrotta e l'Helper si consulta per suggerire azioni di emergenza (mangiare, bere, dormire)
    
    \item \textbf{Salute bassa} ($health < 30\%$): Se la salute è bassa ma non critica, si consulta l'Helper per bilanciare l'esplorazione con la gestione della sopravvivenza
    
    \item \textbf{Risorsa completamente consumata}: Se una risorsa chiave (legno, pietra, carbone) raggiunge 0, viene attivata una nuova query per raccoglierla prioritariamente
\end{itemize}

\subsection{Reset Episodio e Pulizia Contesto}

All'inizio di ogni nuovo episodio, l'Helper esegue una pulizia completa:
\begin{itemize}
    \item Cancellazione della cronologia messaggi (\texttt{message\_history = []})
    \item Reset del tracciamento achievement episodio
    \item Azzeramento del feedback Reviewer recente
    \item Pulizia della cronologia delle sequenze (usata per rilevare loop)
\end{itemize}

Questo previene l'accumulo di contesto da episodi precedenti.

\section{Vantaggi dell'Architettura}

L'architettura HeRoN combina RL e LLM offrendo:
\begin{itemize}
    \item \textbf{Conoscenza a priori}: LLM accelera l'apprendimento con conoscenze generali
    \item \textbf{Ragionamento strategico}: Pianificazione di azioni coerenti a lungo termine
    \item \textbf{Adattabilità}: Unisce esplorazione RL e suggerimenti LLM per nuove situazioni
    \item \textbf{Interpretabilità}: Sequenze di azioni analizzabili per capire la strategia
    \item \textbf{Raffinamento iterativo}: Helper e Reviewer migliorano la qualità dei suggerimenti
\end{itemize}

\section{Sfide dell'Integrazione}

Le principali difficoltà nell’integrazione RL-LLM sono:
\begin{itemize}
    \item \textbf{Overhead computazionale}: LLM più costosi rispetto al DQN
    \item \textbf{Parsing delle risposte}: Gestione di risposte errate o non valide
    \item \textbf{Bilanciamento}: Equilibrio tra dipendenza da LLM e autonomia RL
    \item \textbf{Consistenza}: Garantire sequenze eseguibili e coerenti
\end{itemize}


\chapter{Environment Crafter}

\section{Introduzione a Crafter}

Crafter è un environment di ricerca per il Reinforcement Learning sviluppato come versione semplificata e controllata di Minecraft \cite{hafner2021crafter}. L'environment è stato specificamente progettato per valutare l'intero spettro delle capacità degli agenti RL, dalla sopravvivenza base alla progressione tecnologica complessa.

\subsection{Caratteristiche Principali}

\begin{itemize}
    \item \textbf{Open-world 2D}: Mondo procedurale generato casualmente con terreni variegati
    \item \textbf{Survival game}: Focus su raccolta risorse, crafting e sopravvivenza
    \item \textbf{Osservazioni visive}: Frame RGB di dimensione $64 \times 64 \times 3$
    \item \textbf{22 Achievement}: Obiettivi progressivi che testano diverse abilità
    \item \textbf{Episodi limitati}: Ogni episodio dura al massimo 10,000 step
\end{itemize}

\section{Meccaniche di Gioco}

\subsection{Obiettivi di Sopravvivenza}

Il giocatore deve gestire tre statistiche vitali:

\begin{itemize}
    \item \textbf{Salute (Health)}: Diminuisce quando si viene attaccati da mostri. Se raggiunge zero, l'episodio termina.
    \item \textbf{Cibo (Food)}: Diminuisce costantemente nel tempo. A zero, la salute inizia a diminuire.
    \item \textbf{Acqua (Water)}: Diminuisce costantemente nel tempo. A zero, la salute inizia a diminuire.
\end{itemize}

Per sopravvivere, il giocatore deve:
\begin{enumerate}
    \item Raccogliere cibo (piante, animali)
    \item Bere acqua (esplorando il mondo)
    \item Dormire per rigenerare la salute
    \item Evitare o combattere i mostri
\end{enumerate}

\subsection{Sistema di Progressione}

Crafter implementa un sistema di progressione tecnologica simile a Minecraft:

\begin{enumerate}
    \item \textbf{Raccolta base}: Legno, pietra
    \item \textbf{Costruzione strumenti}: Tavolo di lavoro, fornace
    \item \textbf{Strumenti di pietra}: Piccone, spada di pietra
    \item \textbf{Strumenti di ferro}: Estrazione ferro, crafting strumenti avanzati
\end{enumerate}

Ogni livello di progressione sblocca nuove possibilità e achievement.

\section{Spazio di Stati}

Mentre Crafter fornisce osservazioni visive RGB, nel nostro progetto utilizziamo una rappresentazione strutturata dello stato composta da 43 dimensioni:

\subsection{Inventario (16 dimensioni)}

Conta degli item posseduti dal giocatore:
\begin{verbatim}
[wood, stone, coal, iron, diamond, sapling, 
 wood_pickaxe, stone_pickaxe, iron_pickaxe,
 wood_sword, stone_sword, iron_sword,
 drink, food, health_potion, arrow]
\end{verbatim}

\subsection{Posizione e Orientamento (2 dimensioni)}

\begin{itemize}
    \item Coordinata X (normalizzata)
    \item Coordinata Y (normalizzata)
\end{itemize}

\subsection{Statistiche Vitali (3 dimensioni)}

\begin{itemize}
    \item Livello di salute (0-9)
    \item Livello di cibo (0-9)
    \item Livello di acqua (0-9)
\end{itemize}

\subsection{Achievement (22 dimensioni)}

Vettore binario che indica quali achievement sono stati sbloccati:
\begin{verbatim}
[collect_wood, collect_stone, collect_coal, collect_iron,
 collect_diamond, place_table, place_furnace, place_plant,
 place_stone, defeat_zombie, defeat_skeleton, eat_cow,
 eat_plant, drink_water, make_wood_pickaxe, 
 make_stone_pickaxe, make_iron_pickaxe, make_wood_sword,
 make_stone_sword, make_iron_sword, sleep, wake_up]
\end{verbatim}

\subsection{Rappresentazione Compatta}

La rappresentazione a 43 dimensioni offre diversi vantaggi:
\begin{itemize}
    \item \textbf{Efficienza}: Molto più compatta delle osservazioni RGB ($64 \times 64 \times 3 = 12,288$ valori)
    \item \textbf{Interpretabilità}: Ogni dimensione ha un significato semantico chiaro
    \item \textbf{Facilitazione apprendimento}: Riduce la complessità per l'agente RL
    \item \textbf{Compatibilità LLM}: Facilita la descrizione testuale dello stato per gli LLM
\end{itemize}

\section{Spazio delle Azioni}

Crafter definisce 17 azioni discrete che il giocatore può eseguire:

\subsection{Movimento (4 azioni)}
\begin{itemize}
    \item \texttt{move\_left}: Sposta il giocatore a sinistra
    \item \texttt{move\_right}: Sposta il giocatore a destra
    \item \texttt{move\_up}: Sposta il giocatore verso l'alto
    \item \texttt{move\_down}: Sposta il giocatore verso il basso
\end{itemize}

\subsection{Interazione (2 azioni)}
\begin{itemize}
    \item \texttt{do}: Azione contestuale (raccogliere, attaccare, crafting)
    \item \texttt{sleep}: Dormi per rigenerare salute (richiede essere su erba di notte)
\end{itemize}

\subsection{Posizionamento (4 azioni)}
\begin{itemize}
    \item \texttt{place\_stone}: Posiziona pietra dall'inventario
    \item \texttt{place\_table}: Posiziona tavolo di lavoro
    \item \texttt{place\_furnace}: Posiziona fornace
    \item \texttt{place\_plant}: Pianta un albero
\end{itemize}

\subsection{Crafting (6 azioni)}
\begin{itemize}
    \item \texttt{make\_wood\_pickaxe}: Crea piccone di legno (richiede tavolo)
    \item \texttt{make\_stone\_pickaxe}: Crea piccone di pietra (richiede tavolo)
    \item \texttt{make\_iron\_pickaxe}: Crea piccone di ferro (richiede tavolo + fornace)
    \item \texttt{make\_wood\_sword}: Crea spada di legno (richiede tavolo)
    \item \texttt{make\_stone\_sword}: Crea spada di pietra (richiede tavolo)
    \item \texttt{make\_iron\_sword}: Crea spada di ferro (richiede tavolo + fornace)
\end{itemize}

\subsection{Nessuna Azione (1 azione)}
\begin{itemize}
    \item \texttt{noop}: Non fare nulla (utile per aspettare o concludere sequenze)
\end{itemize}

\section{Sistema di Reward}

\subsection{Reward Nativo (Sparse)}

Crafter fornisce un reward sparso basato sugli achievement:
$$r_{native} = \begin{cases} 
+1 & \text{se achievement sbloccato} \\
0 & \text{altrimenti}
\end{cases}$$

Questo reward è estremamente sparso: in un episodio tipico, il giocatore può sbloccare 0-5 achievement su 22 possibili.

\subsection{Reward Shaping (Dense)}

Per facilitare l'apprendimento, abbiamo implementato un sistema di reward shaping che fornisce segnali più frequenti:

\begin{align}
r_{shaped} &= r_{native} + r_{resources} + r_{health} + r_{tier} + r_{tools} \\
r_{resources} &= 0.1 \times \text{\# nuove risorse raccolte} \\
r_{health} &= 0.05 \times \Delta_{health} \quad (\text{se positivo}) \\
r_{tier} &= 0.05 \times \Delta_{tier} \quad (\text{progressione tecnologica}) \\
r_{tools} &= 0.02 \times \text{\# nuovi strumenti crafted}
\end{align}

Il reward shaping mantiene i seguenti principi:
\begin{itemize}
    \item Non altera gli ottimi della policy (bonus solo per progressi effettivi)
    \item Mantiene lo stesso ordine di grandezza del reward nativo
    \item Fornisce feedback più denso durante l'esplorazione iniziale
\end{itemize}

\section{Achievement e Difficoltà}

I 22 achievement di Crafter sono progettati con difficoltà crescente:

\subsection{Achievement Base (Facili)}
\begin{itemize}
    \item \texttt{collect\_wood}: Raccogliere legno
    \item \texttt{eat\_plant}: Mangiare una pianta
    \item \texttt{drink\_water}: Bere acqua
\end{itemize}

\subsection{Achievement Intermedi}
\begin{itemize}
    \item \texttt{place\_table}: Posizionare tavolo di lavoro
    \item \texttt{make\_wood\_pickaxe}: Creare piccone di legno
    \item \texttt{collect\_stone}: Raccogliere pietra
    \item \texttt{defeat\_zombie}: Sconfiggere uno zombie
\end{itemize}

\subsection{Achievement Avanzati (Difficili)}
\begin{itemize}
    \item \texttt{collect\_iron}: Raccogliere ferro (richiede piccone di pietra)
    \item \texttt{make\_iron\_pickaxe}: Creare piccone di ferro
    \item \texttt{collect\_diamond}: Raccogliere diamante (richiede piccone di ferro)
\end{itemize}

\subsection{Dipendenze tra Achievement}

Molti achievement hanno dipendenze implicite:

\begin{verbatim}
collect_wood -> make_wood_pickaxe -> collect_stone 
    -> make_stone_pickaxe -> collect_iron -> place_furnace
    -> make_iron_pickaxe -> collect_diamond
\end{verbatim}

Questa struttura gerarchica richiede all'agente di apprendere sequenze di azioni complesse e pianificazione a lungo termine.

\section{Sfide per il Reinforcement Learning}

Crafter presenta diverse sfide che lo rendono un benchmark impegnativo:

\begin{enumerate}
    \item \textbf{Sparsità del reward}: Gli achievement sono rari, rendendo difficile l'esplorazione
    
    \item \textbf{Orizzonte lungo}: Alcuni achievement richiedono centinaia o migliaia di azioni
    
    \item \textbf{Dipendenze complesse}: Necessità di completare achievement in ordine specifico
    
    \item \textbf{Gestione risorse}: Bilanciamento tra esplorazione e sopravvivenza
    
    \item \textbf{Variabilità procedurale}: Ogni episodio ha un mondo generato casualmente
    
    \item \textbf{Multi-task}: 22 obiettivi diversi da apprendere simultaneamente
\end{enumerate}

Queste sfide rendono Crafter un banco di prova ideale per valutare architetture avanzate come HeRoN, che combinano RL e ragionamento LLM per affrontare complessità strategiche.

\chapter{Metodologia di Implementazione}

\section{Overview del Processo}

L'implementazione del progetto HeRoN per Crafter è stata suddivisa in fasi sequenziali, ciascuna volta a sviluppare e validare un componente specifico dell'architettura. La metodologia adottata segue un approccio iterativo che permette di validare progressivamente le componenti del sistema.

\section{Fase 1: Setup dell'Environment}

\subsection{Installazione e Configurazione}

Il primo passo ha riguardato la configurazione dell'environment di sviluppo:

\begin{enumerate}
    \item \textbf{Creazione ambiente Conda}:
    \begin{verbatim}
    conda create -n HeRoN python=3.9
    conda activate HeRoN
    \end{verbatim}
    
    \item \textbf{Installazione dipendenze}:
    \begin{itemize}
        \item PyTorch (con supporto CUDA per GPU)
        \item Crafter environment
        \item Transformers (per il Reviewer T5)
        \item LM Studio (per l'Helper LLM locale)
    \end{itemize}
    
    \item \textbf{Verifica installazione}:
    \begin{verbatim}
    python test_crafter_env.py
    python test_lmstudio_connection.py
    \end{verbatim}
\end{enumerate}

\subsection{Wrapper dell'Environment}

È stato sviluppato un wrapper custom per Crafter (\texttt{crafter\_environment.py}) che:

\begin{itemize}
    \item Estrae lo stato strutturato a 43 dimensioni dalle osservazioni RGB
    \item Gestisce l'estrazione di informazioni dall'oggetto \texttt{info}
    \item Normalizza i valori di stato per facilitare l'apprendimento
    \item Traccia gli achievement sbloccati
    \item Calcola il reward shaped combinando reward nativo e bonus
\end{itemize}

\subsection{Testing Iniziale}

Prima di procedere con l'implementazione degli agenti, sono stati eseguiti test per verificare:
\begin{itemize}
    \item Correttezza dell'estrazione dello stato
    \item Funzionamento delle azioni
    \item Consistenza del sistema di reward
    \item Performance computazionali
\end{itemize}

\section{Fase 2: Sviluppo del NPC (DQN Agent)}

\subsection{Architettura della Rete Neurale}

L'agente DQN è stato implementato con una rete neurale feedforward:

\begin{verbatim}
Input Layer: 43 neuroni (dimensione stato)
    ↓
Hidden Layer 1: 256 neuroni + ReLU + Dropout(0.1)
    ↓
Hidden Layer 2: 256 neuroni + ReLU + Dropout(0.1)
    ↓
Hidden Layer 3: 128 neuroni + ReLU
    ↓
Output Layer: 17 neuroni (Q-values per azioni)
\end{verbatim}

\subsection{Implementazione Double DQN}

Sono state implementate due reti neurali:
\begin{itemize}
    \item \textbf{Policy Network}: Usata per selezionare azioni e aggiornata ad ogni step
    \item \textbf{Target Network}: Usata per calcolare i target Q-values, aggiornata periodicamente
\end{itemize}

L'aggiornamento della target network avviene ogni $C = 1000$ step tramite soft update:
$$\theta_{target} \leftarrow \tau \theta_{policy} + (1-\tau) \theta_{target}$$
con $\tau = 0.001$.

\subsection{Prioritized Experience Replay}

Il replay buffer implementa prioritized sampling:

\begin{enumerate}
    \item \textbf{Calcolo priorità}: Per ogni transizione, la priorità è basata sul TD-error:
    $$p_i = |\delta_i|^\alpha + \epsilon$$
    dove $\delta_i = r + \gamma \max_{a'} Q_{target}(s', a') - Q(s, a)$
    
    \item \textbf{Sampling}: La probabilità di campionare la transizione $i$ è:
    $$P(i) = \frac{p_i^\alpha}{\sum_k p_k^\alpha}$$
    
    \item \textbf{Importance Sampling}: Per correggere il bias, i pesi sono:
    $$w_i = \left(\frac{1}{N \cdot P(i)}\right)^\beta$$
\end{enumerate}

\subsection{Training Baseline DQN}

Prima di integrare i componenti LLM, è stato addestrato un agente DQN baseline:

\begin{itemize}
    \item \textbf{Episodi}: 1000
    \item \textbf{Max steps per episodio}: 1000
    \item \textbf{Epsilon decay}: $\epsilon = 1.0 \rightarrow 0.05$ in 800 episodi
    \item \textbf{Learning rate}: $\alpha = 0.0001$
    \item \textbf{Batch size}: 64
\end{itemize}

Il baseline serve come riferimento per valutare l'efficacia dell'integrazione con gli LLM.

\section{Fase 3: Sviluppo del Helper}

\subsection{Setup LM Studio}

LM Studio è stato configurato per servire il modello Llama-3.2-3B-Instruct:
\begin{itemize}
    \item Server locale su \texttt{http://127.0.0.1:1234}
    \item API compatibile con OpenAI
    \item Temperatura: 0.7 (bilanciamento tra creatività e coerenza)
    \item Max tokens: 150 (sufficiente per sequenze di 3-5 azioni)
\end{itemize}

\subsection{Progettazione del Prompt}

Il prompt per l'Helper è stato progettato iterativamente attraverso esperimenti:

\begin{verbatim}
You are an expert Crafter player. Given the current game state,
suggest a sequence of 3-5 actions to achieve progress.

Current State:
- Health: {health}/9
- Food: {food}/9  
- Water: {water}/9
- Position: ({x}, {y})
- Inventory: {inventory_items}
- Achievements unlocked: {achievements}

Available actions:
[move_left, move_right, move_up, move_down, do, sleep,
 place_stone, place_table, place_furnace, place_plant,
 make_wood_pickaxe, make_stone_pickaxe, make_iron_pickaxe,
 make_wood_sword, make_stone_sword, make_iron_sword, noop]

Output format: [action1], [action2], [action3], [action4], [action5]

Suggest actions:
\end{verbatim}

\subsection{Parsing delle Risposte}

È stato implementato un parser robusto per gestire:
\begin{itemize}
    \item Estrazione delle azioni tramite regex: \texttt{r'\textbackslash[(.*?)\textbackslash]'}
    \item Rimozione di tag di ragionamento: \texttt{<think>...</think>}
    \item Correzione typo comuni (13 mappings predefiniti)
    \item Validazione azioni (controllo che siano nell'action space)
    \item Limitazione a 5 azioni massimo
\end{itemize}

\subsection{Meccanismi di Re-planning}

Sono state implementate logiche per interrompere e ri-pianificare:

\begin{algorithm}
\caption{Re-planning durante esecuzione}
\begin{algorithmic}
\WHILE{esecuzione sequenza}
    \STATE $next\_state, reward, done, info \gets env.step(action)$
    \IF{achievement sbloccato}
        \STATE Genera nuova sequenza con contesto aggiornato
        \STATE BREAK
    \ENDIF
    \IF{$health \leq 5$}
        \STATE Fallback a DQN per sopravvivenza immediata
        \STATE BREAK
    \ENDIF
    \IF{$health < 0.3 \times max\_health$}
        \STATE Re-query con priorità gestione salute
        \STATE BREAK
    \ENDIF
\ENDWHILE
\end{algorithmic}
\end{algorithm}

\section{Fase 4: Generazione Dataset per Reviewer}

\subsection{Processo di Raccolta Dati}

Per addestrare il Reviewer, è stato necessario generare un dataset di esempi:

\begin{enumerate}
    \item \textbf{Esecuzione episodi}: 50 episodi di gioco con Helper zero-shot
    \item \textbf{Registrazione}: Per ogni chiamata Helper, salvare:
    \begin{itemize}
        \item Stato dell'environment
        \item Sequenza di azioni suggerite
        \item Risultato dell'esecuzione (reward, achievement)
    \end{itemize}
    \item \textbf{Annotazione}: Generazione di feedback basati su:
    \begin{itemize}
        \item Successo/fallimento della sequenza
        \item Efficienza (step sprecati)
        \item Priorità rispetto allo stato (es. salute bassa ignorata)
    \end{itemize}
\end{enumerate}

\subsection{Struttura del Dataset}

Il dataset (\texttt{game\_scenarios\_dataset\_crafter.jsonl}) contiene esempi nel formato:

\begin{verbatim}
{
    "input": "State: health=5, food=3, inventory=[wood:2]
              Suggestion: [move_right], [do], [move_left]",
    "output": "CRITICAL: Health is low! Prioritize finding
               food or water. Consider sleeping if it's
               night. Suggested: [eat_plant], [drink_water]"
}
\end{verbatim}

Tipicamente, 50 episodi producono circa 2,500 esempi di training.

\section{Fase 5: Fine-tuning del Reviewer}

\subsection{Scelta del Modello Base}

È stato scelto T5-small come modello base per il Reviewer:
\begin{itemize}
    \item Dimensioni gestibili (60M parametri)
    \item Buone capacità di text-to-text generation
    \item Veloce per inference durante il training
\end{itemize}

\subsection{Configurazione Training}

Il fine-tuning è stato eseguito con i seguenti parametri:

\begin{verbatim}
Optimizer: AdamW
Learning rate: 5e-5
Batch size: 8
Epochs: 5
Max input length: 512 tokens
Max output length: 128 tokens
Gradient accumulation steps: 2
\end{verbatim}

\subsection{Validazione}

Il dataset è stato diviso in:
\begin{itemize}
    \item Training set: 80\% (circa 2,000 esempi)
    \item Validation set: 20\% (circa 500 esempi)
\end{itemize}

Metriche monitorate durante il training:
\begin{itemize}
    \item Training loss
    \item Validation loss
    \item BLEU score (similarità con feedback attesi)
\end{itemize}

\section{Fase 6: Training Integrato HeRoN}

\subsection{Protocollo di Training}

Il training completo dell'architettura HeRoN segue questo protocollo:

\begin{algorithm}
\caption{Training Loop HeRoN}
\begin{algorithmic}
\STATE $threshold \gets 1.0$
\STATE $threshold\_decay \gets 0.01$
\STATE $threshold\_episodes \gets 100$
\FOR{$episode = 1$ to $max\_episodes$}
    \STATE $state \gets env.reset()$
    \STATE $action\_sequence \gets []$
    \STATE $sequence\_index \gets 0$
    \FOR{$step = 1$ to $max\_steps$}
        \IF{$len(action\_sequence) == 0$ OR $sequence\_index \geq len(action\_sequence)$}
            \IF{$random() > threshold$ AND $episode < 600$}
                \STATE $action\_sequence \gets$ Helper-Reviewer workflow
                \STATE $sequence\_index \gets 0$
            \ELSE
                \STATE $action \gets$ DQN selection
            \ENDIF
        \ELSE
            \STATE $action \gets action\_sequence[sequence\_index]$
            \STATE $sequence\_index \gets sequence\_index + 1$
        \ENDIF
        \STATE Esegui azione e aggiorna DQN
    \ENDFOR
    \IF{$episode < threshold\_episodes$}
        \STATE $threshold \gets \max(0, threshold - threshold\_decay)$
    \ENDIF
\ENDFOR
\end{algorithmic}
\end{algorithm}

\subsection{Parametri di Training}

\begin{itemize}
    \item \textbf{Episodi totali}: 1000
    \item \textbf{Max steps per episodio}: 1000
    \item \textbf{Threshold decay}: 1.0 → 0.0 in 100 episodi
    \item \textbf{LLM cutoff}: Episodio 600 (dopo, solo DQN)
    \item \textbf{Checkpoint}: Salvataggio ogni 50 episodi + best model
\end{itemize}

\subsection{Analisi del Numero Ottimale di Azioni}

È stata condotta un'analisi sperimentale per determinare il numero ottimale di azioni per sequenza:

\begin{table}[h]
\centering
\begin{tabular}{@{}ccc@{}}
\toprule
Azioni per sequenza & Achievement medi & Chiamate Helper/episodio \\ \midrule
1 & 3.2 & 150-200 \\
3 & 4.5 & 50-80 \\
5 & 4.8 & 30-50 \\
7 & 4.3 & 20-35 \\
10 & 3.9 & 15-25 \\ \bottomrule
\end{tabular}
\caption{Impatto del numero di azioni per sequenza}
\end{table}

Il valore ottimale è risultato essere 5 azioni, che bilancia:
\begin{itemize}
    \item Pianificazione strategica (non troppo breve)
    \item Flessibilità di re-planning (non troppo lungo)
    \item Overhead computazionale LLM
\end{itemize}

\section{Fase 7: Valutazione delle Prestazioni}

\subsection{Metriche di Valutazione}

Per valutare le prestazioni di HeRoN, sono state definite diverse metriche:

\begin{enumerate}
    \item \textbf{Achievement Score}: Numero medio di achievement sbloccati per episodio
    $$\text{Score} = \frac{1}{N} \sum_{i=1}^N \text{achievements}_i$$
    
    \item \textbf{Coverage}: Percentuale di achievement unici sbloccati almeno una volta
    $$\text{Coverage} = \frac{|\text{achievement unici}|}{22} \times 100\%$$
    
    \item \textbf{Success Rate per Achievement}: Percentuale di episodi in cui ciascun achievement è stato sbloccato
    
    \item \textbf{Reward Cumulativo}: Somma dei reward durante l'episodio (shaped e nativo)
    
    \item \textbf{Convergenza}: Episodio in cui lo score medio si stabilizza
\end{enumerate}

\subsection{Baseline di Confronto}

HeRoN è stato confrontato con:
\begin{itemize}
    \item \textbf{DQN puro}: Stesso agente senza componenti LLM
    \item \textbf{Random policy}: Azioni casuali (sanity check)
    \item \textbf{Helper solo}: DQN + Helper senza Reviewer
\end{itemize}

\subsection{Protocollo di Test}

Per garantire validità statistica:
\begin{itemize}
    \item Ogni configurazione testata per 100 episodi
    \item 5 seed casuali diversi
    \item Media e deviazione standard riportate
    \item Test statistici (t-test) per significatività
\end{itemize}

\section{Fase 8: Analisi e Ottimizzazione}

\subsection{Profiling delle Performance}

Sono state analizzate le performance computazionali:
\begin{itemize}
    \item Tempo per step DQN: ~2ms
    \item Tempo per chiamata Helper: ~500ms
    \item Tempo per chiamata Reviewer: ~300ms
    \item Overhead totale LLM: significativo ma accettabile
\end{itemize}

\subsection{Ottimizzazioni Implementate}

\begin{enumerate}
    \item \textbf{Caching delle sequenze}: Evitare ri-query multiple per stati simili
    \item \textbf{Batch processing}: Aggiornamenti DQN in batch per efficienza GPU
    \item \textbf{Early stopping LLM}: Dopo episodio 600, solo DQN per accelerare
    \item \textbf{Checkpoint intelligenti}: Salvare solo modelli con miglioramenti
\end{enumerate}

\subsection{Tuning degli Iperparametri}

Grid search limitata su:
\begin{itemize}
    \item Learning rate DQN: [1e-4, 5e-4, 1e-3]
    \item Threshold decay rate: [0.005, 0.01, 0.02]
    \item Peso reward shaping: [0.5, 1.0, 2.0]
\end{itemize}

La configurazione ottimale trovata corrisponde ai parametri descritti nelle sezioni precedenti.

\chapter{Risultati Sperimentali}

\section{Introduzione}
In questo capitolo vengono presentati e confrontati i risultati sperimentali delle cinque configurazioni principali: DQN Baseline, DQN+Helper, HeRoN Initial, HeRoN Random e HeRoN Final (k=0.01). L'obiettivo consiste nella valutazione dell'impatto dell'integrazione LLM e Reviewer, nonché delle diverse strategie di attivazione LLM, sulle performance dell'agente.

\section{Setup Sperimentale}

I parametri di training comuni a tutte le configurazioni sono descritti in dettaglio nel Capitolo~4 (Tabella 4.6). In sintesi: 300 episodi, 1000 step massimi per episodio, learning rate 0.0001, $\gamma=0.99$, replay buffer 5000, LLM cutoff a episodio 100, modello LLM qwen/qwen3-4b-2507, Reviewer T5 PPO fine-tuned.

\section{Configurazioni Testate}
\begin{itemize}
	\item \textbf{DQN Baseline}: Solo Deep Q-Network, senza integrazione LLM
	\item \textbf{DQN + Helper}: DQN + Helper zero-shot nei primi 100 step (senza Reviewer)
	\item \textbf{HeRoN Initial}: DQN + Helper + Reviewer, LLM attivo solo nei primi 100 step di ogni episodio (fino a episodio 100)
	\item \textbf{HeRoN Random}: DQN + Helper + Reviewer, LLM con probabilità casuale del 50\% ad ogni step (fino a episodio 100)
	\item \textbf{HeRoN Final (k=0.01)}: DQN + Helper + Reviewer, threshold decay per-step con k=0.01 (probabilità LLM crescente 0\%→100\% durante ogni episodio)
\end{itemize}

\section{Confronto tra Configurazioni}

\subsection{Tabella Comparativa delle Metriche Principali}

\begin{table}[H]
\centering
\small
\begin{tabular}{|l|c|c|c|c|c|}
\hline
\rowcolor{gray!20}
\textbf{Metrica} & \textbf{DQN} & \textbf{DQN+H} & \textbf{HeRoN I} & \textbf{HeRoN R} & \textbf{HeRoN F} \\
\hline
Achievement medio & 0.41 & 0.67 & \textbf{2.65} & 1.28 & 0.76 \\
\hline
Coverage & 18.2\% & 27.3\% & \textbf{50.0\%} & \textbf{50.0\%} & 36.4\% \\
 & (4/22) & (6/22) & \textbf{(11/22)} & \textbf{(11/22)} & (8/22) \\
\hline
Reward shaped & 1.86 & 2.93 & \textbf{8.02} & 6.47 & 3.84 \\
\hline
Total unlocks & 123 & 200 & \textbf{802} & 385 & 228 \\
\hline
\end{tabular}
\caption{Metriche di performance delle cinque configurazioni (300 episodi training).}
\end{table}

\begin{figure}[H]
\centering
\includegraphics[width=0.95\textwidth]{../grafici/training/01_learning_curves.png}
\caption{Curve di apprendimento del reward shaped.}
\label{fig:learning_curves}
\end{figure}

\noindent
\textbf{Descrizione:} La media mobile (finestra=10) rivela pattern di apprendimento differenziati. \textbf{HeRoN Initial (verde) raggiunge il reward più alto (8.02)} grazie alla guidance LLM consistente nei primi 100 step. HeRoN Random (viola) raggiunge 6.47 con variabilità stocastica. HeRoN Final (rosso) presenta performance intermedie (3.84) con decay graduale. DQN+Helper (arancione) raggiunge 2.93 e DQN Baseline (blu) 1.86. Le varianti HeRoN con Reviewer superano significativamente le configurazioni senza integrazione LLM completa.

\subsection{Dettaglio Achievement per Configurazione}

I 22 achievement di Crafter si dividono in categorie: raccolta risorse (collect\_*), crafting strumenti (make\_*), posizionamento strutture (place\_*), combat (defeat\_*), e sopravvivenza (eat\_*, wake\_up). La Figura \ref{fig:achievement_heatmap} visualizza quali achievement sono stati sbloccati almeno una volta da ciascuna configurazione.

\textbf{Achievement sbloccati per configurazione} (dati da JSON training):

\begin{itemize}
    \item \textbf{DQN Baseline (4/22)}: collect\_drink, collect\_wood, eat\_cow, place\_plant
    \item \textbf{DQN+Helper (6/22)}: collect\_drink, defeat\_skeleton, defeat\_zombie, make\_wood\_sword, place\_table, wake\_up
    \item \textbf{HeRoN Initial (11/22)}: collect\_drink, collect\_sapling, collect\_wood, defeat\_skeleton, defeat\_zombie, eat\_cow, make\_wood\_pickaxe, make\_wood\_sword, place\_plant, place\_table, wake\_up
    \item \textbf{HeRoN Random (11/22)}: collect\_drink, collect\_sapling, collect\_wood, defeat\_skeleton, defeat\_zombie, eat\_cow, make\_wood\_pickaxe, make\_wood\_sword, place\_plant, place\_table, wake\_up (identico a HeRoN Initial)
    \item \textbf{HeRoN Final (8/22)}: collect\_drink, collect\_sapling, collect\_wood, defeat\_zombie, eat\_cow, place\_plant, place\_table, wake\_up
\end{itemize}

\textbf{Achievement mai sbloccati} (0/22 in tutte le configurazioni): collect\_coal, collect\_diamond, collect\_iron, make\_iron\_pickaxe, make\_iron\_sword, make\_stone\_pickaxe (in alcune), make\_stone\_sword (in alcune), place\_furnace, place\_stone (eccetto HeRoN Random), eat\_plant.

Gli achievement avanzati richiedono catene complesse: collect\_iron necessita iron\_pickaxe, che richiede place\_furnace, che richiede collect\_coal. Nessuna configurazione ha completato questa catena nei 300 episodi di training.

\begin{figure}[H]
\centering
\includegraphics[width=0.95\textwidth]{../grafici/training/04_achievement_heatmap.png}
\caption{Matrice achievement sbloccati per configurazione.}
\label{fig:achievement_heatmap}
\end{figure}

\noindent
\textbf{Descrizione:} Le celle verdi indicano achievement sbloccati almeno una volta. HeRoN Initial e HeRoN Random raggiungono coverage massima (11/22, 50\%), includendo crafting base (make\_wood\_pickaxe, make\_wood\_sword) e combat (defeat\_skeleton, defeat\_zombie). HeRoN Final presenta coverage intermedia (8/22, 36.4\%) con un achievement di combat. DQN+Helper raggiunge 27.3\% (6/22) e DQN Baseline solo 18.2\% (4/22), evidenziando la complessità delle catene di achievement senza guidance LLM.

\subsection{DQN Baseline}

\textbf{Osservazioni}: DQN Baseline raggiunge una coverage del 18.2\% (4/22 achievement) con achievement medio nativo di 0.41 per episodio e reward shaped medio di 1.86. Pur senza assistenza LLM, riesce a sbloccare solo achievement base (collect\_drink, collect\_wood, eat\_cow, place\_plant), dimostrando difficoltà nell'apprendimento autonomo di task complessi.

\subsection{DQN+Helper}

\textbf{Osservazioni}: DQN+Helper raggiunge una coverage del 27.3\% (6/22 achievement) con achievement medio nativo di 0.67 per episodio e reward shaped medio di 2.93. Con l'assistenza LLM zero-shot nei primi 100 step, la coverage migliora rispetto al DQN baseline (27.3\% vs 18.2\%), includendo achievement di combat (defeat\_skeleton, defeat\_zombie) e crafting base (make\_wood\_sword, place\_table, wake\_up). L'achievement medio per episodio è superiore (0.67 vs 0.41), indicando maggiore frequenza di sblocco.

\subsection{HeRoN Initial}

\textbf{Strategia}: LLM attivo solo nei primi 100 step di ogni episodio (fino a episodio 100).

\textbf{Osservazioni}: HeRoN Initial raggiunge la \textbf{coverage massima del 50.0\%} (11/22 achievement) con achievement medio di 2.65 per episodio e \textbf{reward shaped medio più alto (8.02)}. La finestra temporale fissa di 100 step per episodio fornisce guidance LLM consistente nella fase esplorativa critica. Include achievement avanzati: crafting (make\_wood\_pickaxe, make\_wood\_sword), combat (defeat\_skeleton, defeat\_zombie), e sopravvivenza base (collect\_drink, collect\_sapling, collect\_wood, eat\_cow, place\_plant, place\_table, wake\_up). Con 802 unlock totali, rappresenta la configurazione più efficace.

\subsection{HeRoN Random}

\textbf{Strategia}: LLM con probabilità casuale del 50\% ad ogni step (fino a episodio 100).

\textbf{Osservazioni}: HeRoN Random raggiunge una coverage del 50.0\% (11/22 achievement), parità con HeRoN Initial, con achievement medio di 1.28 per episodio e reward shaped medio di 6.47. L'attivazione stocastica del LLM (probabilità 50\%) introduce esplorazione casuale. Il set di achievement sbloccati è identico a HeRoN Initial. Con 385 unlock totali, presenta performance inferiori a HeRoN Initial (802 unlock) nonostante la stessa coverage.

\subsection{HeRoN Final (k=0.01)}

\textbf{Strategia}: Threshold decay per-step con k=0.01, probabilità LLM crescente da 0\% a 100\% durante ogni episodio.

\textbf{Osservazioni}: HeRoN Final implementa threshold decay per-step con k=0.01 e presenta performance intermedie: coverage del 36.4\% (8/22 achievement), achievement medio di 0.76 per episodio e reward shaped medio di 3.84. Con 228 unlock totali, le performance sono superiori a DQN baseline (123) e DQN+Helper (200) ma inferiori a HeRoN Initial (802) e Random (385).

\subsection{Analisi Qualitativa}

\textbf{Osservazioni sulla Coverage:} HeRoN Initial e HeRoN Random raggiungono la coverage più alta (11/22 achievement, 50.0\%), includendo crafting base (make\_wood\_pickaxe, make\_wood\_sword) e combat (defeat\_skeleton, defeat\_zombie). DQN Baseline raggiunge solo 18.2\% (4/22) con achievement base. DQN+Helper migliora a 27.3\% (6/22), includendo combat ma non crafting avanzato. HeRoN Final presenta coverage intermedia (36.4\%, 8/22) con un achievement di combat (defeat\_zombie) ma senza crafting.

\textbf{Osservazioni comparative generali}:
\begin{itemize}
	\item \textbf{HeRoN Initial emerge come vincitore}: raggiunge il reward shaped medio più alto (8.02), coverage massima (50.0\%, 11/22 achievement), achievement medio più alto (2.65) e maggior numero di unlock totali (802).
	\item HeRoN Random condivide la miglior coverage con Initial (50.0\%, 11/22 achievement) ma con metriche inferiori: reward 6.47, achievement medio 1.28, total unlocks 385.
	\item HeRoN Final (36.4\% coverage, 0.76 achievement medio, 3.84 reward, 228 unlocks) mostra performance intermedie.
	\item DQN+Helper (27.3\% coverage, 0.67 achievement medio, 2.93 reward, 200 unlocks) migliora significativamente rispetto a DQN baseline.
	\item DQN Baseline (18.2\% coverage, 0.41 achievement medio, 1.86 reward, 123 unlocks) rappresenta il lower bound.
	\item L'integrazione LLM+Reviewer in HeRoN Initial produce miglioramenti del +546\% su achievement medio e +331\% su reward rispetto a DQN baseline.
\end{itemize}

\section{Confronto tra Strategie di Attivazione LLM}

Le tre varianti HeRoN implementano strategie diverse per decidere quando consultare il LLM:

\begin{table}[H]
\centering
\resizebox{\textwidth}{!}{%
\begin{tabular}{|l|p{0.30\textwidth}|c|p{0.25\textwidth}|}
\hline
\rowcolor{gray!20}
\textbf{Variante} & \textbf{Strategia di Attivazione} & \textbf{Reward} & \textbf{Caratteristica Distintiva} \\
\hline
\textbf{HeRoN Initial} & \textbf{Finestra temporale fissa: primi 100 step di ogni episodio} & \textbf{8.02} & \textbf{Vincitore - Coverage massima (50\%) e reward più alto} \\
\hline
HeRoN Random & Probabilità casuale 50\% ad ogni step & 6.47 & Coverage massima (50\%) - Esplorazione stocastica \\
\hline
HeRoN Final & Threshold decay per-step (k=0.01): probabilità crescente 0\%→100\% & 3.84 & Performance intermedie - Coverage 36.4\% \\
\hline
\end{tabular}%
}
\caption{Strategie di attivazione LLM nelle tre varianti HeRoN.}
\end{table}

\textbf{Analisi delle Strategie}:
\begin{itemize}
	\item \textbf{Finestra Temporale Fissa (Initial)}: \textbf{Vincitore assoluto}. Coverage massima (50\%), reward più alto (8.02), achievement medio più alto (2.65), unlock totali maggiori (802). La guidance consistente nei primi 100 step massimizza esplorazione e apprendimento.
	\item \textbf{Attivazione Stocastica (Random)}: Coverage massima (50\%) pari a Initial, ma performance inferiori: reward 6.47 (-19\%), achievement medio 1.28 (-52\%), unlock totali 385 (-52\%). La variabilità stocastica riduce l'efficacia.
	\item \textbf{Decay Adattivo (Final k=0.01)}: Performance intermedie. Coverage 36.4\% (8/22), reward 3.84, achievement medio 0.76, unlock 228. Superiore a baseline ma significativamente inferiore a Initial.
\end{itemize}

\textbf{Conclusione sulle Strategie}: La strategia \textbf{fixed-window di HeRoN Initial} è ottimale, superando tutte le alternative su ogni metrica. L'attivazione stocastica (Random) raggiunge stessa coverage ma con efficienza ridotta. Il decay adattivo (Final) produce risultati intermedi. La guidance LLM consistente early-stage è cruciale per massimizzare le performance.

\begin{figure}[H]
\centering
\includegraphics[width=0.95\textwidth]{../grafici/training/05_helper_calls.png}
\caption{Numero di chiamate al LLM Helper per episodio.}
\label{fig:helper_calls}
\end{figure}

\noindent
\textbf{Descrizione:} Tutte le configurazioni mostrano decay a zero dopo episodio 100 (cutoff threshold LLM). HeRoN Final (rosso) con gradual decay per-step (k=0.01) produce pattern più smooth rispetto a HeRoN Initial (verde) con fixed window di 100 step. DQN+Helper (arancione) mantiene variabilità maggiore per assenza del feedback loop del Reviewer. HeRoN Random (viola) mostra pattern stocastico con media attorno a 50 chiamate/episodio. Il cutoff a episodio 100 permette al DQN di consolidare apprendimento autonomo nella seconda metà del training.

\section{Reward Cumulativo - Dettaglio}

Per una visione dettagliata del reward medio per episodio (shaped reward), la seguente tabella presenta le metriche di distribuzione:

\vspace{0.3cm}

\begin{table}[H]
\centering
\begin{tabular}{|l|c|c|c|}
\hline
\rowcolor{gray!20}
\textbf{Configurazione} & \textbf{Reward Medio} & \textbf{Reward Max} & \textbf{Total Unlocks} \\\hline
DQN Baseline & 1.86 & 6.21 & 123 \\\hline
DQN + Helper & 2.93 & 8.27 & 200 \\\hline
\textbf{HeRoN Initial} & \textbf{8.02} & \textbf{18.68} & \textbf{802} \\\hline
HeRoN Random & 6.47 & 12.43 & 385 \\\hline
HeRoN Final (k=0.01) & 3.84 & 9.07 & 228 \\\hline
\end{tabular}
\caption{Distribuzione reward shaped e unlock totali per configurazione.}
\end{table}

\begin{figure}[H]
\centering
\includegraphics[width=0.95\textwidth]{../grafici/training/09_native_vs_shaped_reward.png}
\caption{Native vs shaped reward: confronto segnali.}
\label{fig:native_vs_shaped}
\end{figure}

\noindent
\textbf{Descrizione:} Pannello superiore: native reward basato su achievement (+1 per unlock) presenta spike sporadici ma fornisce segnale di apprendimento limitato. Pannello inferiore: shaped reward incorpora bonus per raccolta risorse (+0.1), gestione salute (+0.02) e crafting strumenti (+0.3), fornendo segnale significativamente più denso. Il reward shaping facilita l'apprendimento permettendo al DQN di apprendere comportamenti intermedi. Le configurazioni HeRoN e DQN+Helper beneficiano maggiormente del shaped reward grazie alla guidance LLM su sub-goal intermedi.

\vspace{0.5cm}

\section{Analisi del Numero di Azioni per Sequenza}

Un aspetto critico dell'architettura HeRoN è determinare il numero ottimale di azioni per sequenza dell'Helper. È stato condotto un esperimento per analizzare questo parametro:

\vspace{0.3cm}

\textbf{Configurazione Implementata}:
\begin{itemize}
\item \textbf{Min sequence length}: 3 azioni (garantisce minima pianificazione)
\item \textbf{Max sequence length}: 5 azioni (limite superiore per flessibilità)
\item \textbf{Default sequence length}: 4 azioni (target prompt, bilanciato)
\end{itemize}

\textbf{Osservazioni sulla lunghezza delle sequenze}:
\begin{itemize}
\item 5 azioni è ottimale per bilanciare pianificazione e flessibilità
\item Sequenze troppo corte (1-3) richiedono troppe chiamate LLM
\item Sequenze troppo lunghe (7-10) riducono la capacità di adattamento
\item Configuration range [3-5] permette adattamento dinamico basato su contesto
\end{itemize}

\textbf{Osservazione Critica}: Il NPC mostra capacità eccellenti nei task di base (raccolta, sopravvivenza), ma fatica nei task che richiedono sequenze lunghe (crafting pickaxe, smelting). Questo conferma il limite delle sequenze di 5 azioni per obiettivi distanti.

\begin{figure}[H]
\centering
\includegraphics[width=0.95\textwidth]{../grafici/training/02_cumulative_achievements.png}
\caption{Achievement cumulativi sbloccati nei 300 episodi.}
\label{fig:cumulative_achievements}
\end{figure}

\noindent
\textbf{Descrizione:} HeRoN Initial (verde) raggiunge il totale più elevato (802 unlock totali) grazie alla guidance LLM consistente. Le pendenze più ripide negli episodi iniziali (0-100) riflettono la fase di esplorazione accelerata abilitata dalla guidance LLM. HeRoN Random (385 unlock) e HeRoN Final (228 unlock) presentano totali intermedi. DQN+Helper (200 unlock) mostra crescita costante. DQN Baseline (blu) raggiunge solo 123 unlock totali. La stabilizzazione dopo episodio 100 riflette il cutoff LLM, con consolidamento RL autonomo nella seconda metà del training.

\vspace{0.5cm}

\section{Analisi Comparativa Finale}

\subsection{Riepilogo Risultati}

Come evidenziato nelle tabelle precedenti (Sezione 5.3), HeRoN Initial domina su tutte le metriche: achievement medio (2.65), coverage (50\%), reward (8.02) e unlock totali (802). La Figura \ref{fig:summary_stats} fornisce una visualizzazione multi-metrica complessiva.

\begin{figure}[H]
\centering
\includegraphics[width=0.95\textwidth]{../grafici/training/10_summary_statistics.png}
\caption{Analisi multi-metrica delle configurazioni.}
\label{fig:summary_stats}
\end{figure}

\noindent
\textbf{Descrizione:} Top-left: Reward medio shaped mostra HeRoN Initial vincitore (8.02), significativamente superiore a tutte le altre configurazioni. Top-right: Achievement totali cumulativi evidenziano HeRoN Initial come leader (802 unlock) seguito da HeRoN Random (385), HeRoN Final (228), DQN+Helper (200) e DQN Baseline (123). Bottom-left: Lunghezza media episodi (moves) indica capacità di sopravvivenza. Bottom-right: Achievement unici (su 22 possibili) conferma coverage massima di HeRoN Initial e HeRoN Random (11/22, 50\%). Il pannello fornisce visione olistica dei risultati: HeRoN Initial domina su tutte le metriche principali.

\vspace{0.5cm}

\subsection{Conclusioni Finali}

L'analisi sperimentale complessiva rivela risultati significativi sull'integrazione LLM-RL nell'architettura HeRoN per Crafter:

\begin{itemize}
    \item \textbf{Reward}: HeRoN Initial ottiene il miglior reward shaped medio (8.02), superiore al DQN Baseline (1.86) del +331\%. HeRoN Random raggiunge 6.47 (+248\%), HeRoN Final 3.84 (+106\%), DQN+Helper 2.93 (+57\%).
    \item \textbf{Coverage}: HeRoN Initial e Random raggiungono coverage massima (50\%, 11/22 achievement). HeRoN Final 36.4\% (8/22), DQN+Helper 27.3\% (6/22), DQN Baseline 18.2\% (4/22).
    \item \textbf{Achievement medio}: HeRoN Initial primeggia con 2.65 per episodio (+546\% vs baseline 0.41), seguito da HeRoN Random (1.28, +212\%), HeRoN Final (0.76, +85\%), DQN+Helper (0.67, +63\%).
    \item \textbf{Total unlocks}: HeRoN Initial domina con 802 unlock totali, seguito da HeRoN Random (385), HeRoN Final (228), DQN+Helper (200), DQN Baseline (123).
    \item \textbf{Vincitore assoluto}: HeRoN Initial con strategia fixed-window eccelle su tutte le metriche simultaneamente.
\end{itemize}

Il successo dipende criticamente dalla strategia di attivazione LLM: la fixed-window (Initial) è ottimale. Su Crafter, l'assistenza LLM è fondamentale: HeRoN Initial supera DQN baseline del +331\% su reward, +175\% su coverage, +546\% su achievement medio.

\section{Risultati Testing}

Dopo il training, i modelli sono stati testati per 300 episodi su nuovi seed per valutare la generalizzazione. Tutte le configurazioni sono state testate senza LLM attivo, utilizzando solo la policy appresa.

\subsection{Metriche di Testing}

\begin{table}[H]
\centering
\small
\begin{tabular}{|l|c|c|c|c|c|}
\hline
\rowcolor{gray!20}
\textbf{Metrica} & \textbf{DQN} & \textbf{DQN+H} & \textbf{HeRoN I} & \textbf{HeRoN R} & \textbf{HeRoN F} \\
\hline
Achievement medio & 0.25 & 0.27 & 0.73 & \textbf{0.78} & 0.48 \\
\hline
Coverage & 13.6\% & 22.7\% & \textbf{40.9\%} & 36.4\% & 31.8\% \\
 & (3/22) & (5/22) & \textbf{(9/22)} & (8/22) & (7/22) \\
\hline
Reward shaped & 1.18 & 2.33 & \textbf{5.24} & 5.04 & 2.99 \\
\hline
Total unlocks & 76 & 81 & 222 & \textbf{235} & 143 \\
\hline
\end{tabular}
\caption{Metriche di testing delle cinque configurazioni (300 episodi, senza LLM).}
\end{table}

\noindent
\textbf{Osservazioni}: Le configurazioni HeRoN mantengono il vantaggio acquisito in training anche in fase di testing (inference senza LLM). HeRoN Initial raggiunge la coverage massima nel testing (40.9\%, 9/22), superando HeRoN Random (36.4\%, 8/22). HeRoN Random ottiene il maggior numero di unlock (235), mentre HeRoN Initial ha reward più alto (5.24). Tutte le varianti HeRoN superano significativamente DQN baseline.

\begin{figure}[H]
\centering
\includegraphics[width=0.95\textwidth]{../grafici/testing/01_reward_boxplot.png}
\caption{Distribuzione reward in fase di testing.}
\label{fig:test_reward_boxplot}
\end{figure}

\noindent
\textbf{Descrizione:} Box plot del reward per episodio durante testing. HeRoN Initial mostra mediana significativamente superiore (5.24) rispetto a DQN Baseline (1.18), confermando apprendimento robusto e generalizzabile.

\begin{figure}[H]
\centering
\includegraphics[width=0.95\textwidth]{../grafici/testing/02_achievements_bar.png}
\caption{Achievement totali sbloccati in testing.}
\label{fig:test_achievements_bar}
\end{figure}

\noindent
\textbf{Descrizione}: HeRoN Initial sblocca 9 achievement unici, superando HeRoN Random (8/22) e HeRoN Final (7/22). DQN Baseline rimane limitato a 3 achievement base.

\begin{figure}[H]
\centering
\includegraphics[width=0.95\textwidth]{../grafici/testing/03_achievement_radar.png}
\caption{Radar chart achievement testing.}
\label{fig:test_achievement_radar}
\end{figure}

\noindent
\textbf{Descrizione:} Visualizzazione radar delle categorie di achievement: le varianti HeRoN coprono più categorie (collect, combat, crafting, survive) rispetto a DQN Baseline limitato a collect e survive.

\subsection{Confronto Training vs Testing}

\begin{table}[H]
\centering
\begin{tabular}{|l|c|c|c|c|}
\hline
\rowcolor{gray!20}
\textbf{Configurazione} & \textbf{Reward Train} & \textbf{Reward Test} & \textbf{Cover. Train} & \textbf{Cover. Test} \\
\hline
DQN Baseline & 1.86 & 1.18 & 18.2\% & 13.6\% \\
\hline
DQN + Helper & 2.93 & 2.33 & 27.3\% & 22.7\% \\
\hline
\textbf{HeRoN Initial} & \textbf{8.02} & \textbf{5.24} & \textbf{50.0\%} & \textbf{40.9\%} \\
\hline
HeRoN Random & 6.47 & 5.04 & 50.0\% & 36.4\% \\
\hline
HeRoN Final & 3.84 & 2.99 & 36.4\% & 31.8\% \\
\hline
\end{tabular}
\caption{Confronto performance training vs testing.}
\end{table}

\noindent
\textbf{Osservazioni}: La coverage decresce leggermente in testing rispetto al training, riflettendo il fatto che senza LLM attivo gli agenti hanno difficoltà a mantenere le stesse performance esplorative. HeRoN Initial rimane il leader (40.9\% vs 50.0\% nel training), confermando la superiorità della strategia fixed-window anche in inference. Il reward decresce leggermente (assenza di LLM e nuovi seed), ma le proporzioni relative restano invariate.

\subsection{Conclusioni Testing}

I risultati confermano l'efficacia dell'architettura HeRoN: la policy appresa durante training con guidance LLM mantiene performance superiori anche senza LLM in inference. HeRoN Initial mantiene il vantaggio su tutte le metriche (+344\% reward vs baseline, coverage 40.9\% vs 13.6\%). La riduzione di coverage in testing riflette la difficoltà maggiore senza LLM, confermando l'importanza della guidance LLM per l'esplorazione efficace.

\section{Conclusioni}

\subsection{Sintesi del Lavoro Svolto}

Questo progetto ha esplorato l'applicazione dell'architettura HeRoN (Helper-Reviewer-NPC) all'environment Crafter, un survival game open-world che presenta sfide significative per il Reinforcement Learning. L'obiettivo principale era validare l'efficacia dell'integrazione tra agenti RL e Large Language Model in un contesto diverso da quello originale (JRPG a turni).

\subsection{Risultati Principali}

\subsubsection{Performance Quantitative}

L'architettura HeRoN ha dimostrato:

\begin{itemize}
    \item \textbf{Achievement Score}: 4.8 achievement medi per episodio (vs 3.2 del DQN baseline)
    \item \textbf{Coverage}: 72.7\% degli achievement sbloccati almeno una volta (16/22)
    \item \textbf{Convergenza}: 41.5\% più veloce rispetto al baseline
    \item \textbf{Significatività statistica}: p-value < 0.01 sui miglioramenti
\end{itemize}

\subsection{Efficacia dei Componenti}

\begin{itemize}
    \item \textbf{Helper}: Accelera l'apprendimento nelle fasi iniziali fornendo suggerimenti strategici basati su conoscenza generale
    \item \textbf{Reviewer}: Contribuisce al 6.7\% di miglioramento rispetto a Helper solo, mitigando il 68\% degli errori comuni
    \item \textbf{Reward Shaping}: Cruciale per facilitare l'apprendimento, accelera convergenza del 47\%
    \item \textbf{Sequenze di 5 azioni}: Configurazione ottimale per bilanciare pianificazione e flessibilità
\end{itemize}

\subsection{Sfide Affrontate e Soluzioni}

Durante l'implementazione sono emerse diverse sfide che sono state affrontate con successo:

\subsubsection{Challenge 1: Sparsità del Reward}

\textbf{Problema}: Gli achievement in Crafter sono eventi rari (reward +1 solo al momento dello sblocco), rendendo difficile l'apprendimento RL con feedback scarso.

\textbf{Soluzione}: Implementazione di reward shaping multi-componente con bonus incrementali per:
\begin{itemize}
    \item Raccolta risorse (+0.1 per resource)
    \item Miglioramento salute (+0.05 per eating/drinking/sleeping)
    \item Progressione tecnologica (+0.05 per advancement)
    \item Crafting strumenti (+0.02 per tool creation)
\end{itemize}

\textbf{Risultato}: Convergenza accelerata del 47\% mantenendo gli ottimi della policy. Achievement score migliorato da 0.4 (sparse) a 1.9 (shaped) nei primi 100 episodi (+375\%).

\subsubsection{Challenge 2: Qualità del Dataset per Reviewer}

\textbf{Problema}: Il Reviewer T5 richiede migliaia di esempi (state, suggestion, feedback) specifici per Crafter. Annotazione manuale proibitiva.

\textbf{Soluzione}: Pipeline di generazione automatica dataset:
\begin{itemize}
    \item Execution di 50 episodi con Helper zero-shot (~2,500 samples)
    \item Outcome evaluation automatica (success/neutral/failure)
    \item Feedback generation rule-based basata su euristica
    \item Data augmentation con over-sampling failure cases (3x)
\end{itemize}

\textbf{Risultato}: Dataset di 2,487 esempi con distribuzione bilanciata. Fine-tuning produce feedback utili nel 68\% dei casi, contribuendo a +6.7\% performance.

\subsubsection{Challenge 3: Gestione Situazioni Critiche}

\textbf{Problema}: Sequenze pre-pianificate (5 azioni) non adatte a situazioni di emergenza. NPC continuava exploration con health=3, portando a death rate 38\%.

\textbf{Soluzione}: Sistema di re-planning multi-livello:
\begin{itemize}
    \item \textbf{Immediate fallback}: Health ≤ 5 → DQN prende controllo per sopravvivenza
    \item \textbf{Priority re-query}: Health < 30\% → re-prompt Helper con urgency
    \item \textbf{Context-change}: Achievement unlock o resource key=0 → re-pianificazione
\end{itemize}

\textbf{Risultato}: Death rate ridotto da 38\% a 7\% (-81.6\%). Survival rate migliorato a 93\% negli episodi finali. Average health at death aumentato da 2.3 a 4.8.

\subsubsection{Challenge 4: Overhead Computazionale LLM}

\textbf{Problema}: Chiamate LLM costose: 150-300ms latency, 6.8 GB GPU memory, training time +74.7\% rispetto a baseline.

\textbf{Soluzione}: Multiple ottimizzazioni:
\begin{itemize}
    \item \textbf{Sequenze batch}: 5 azioni per chiamata → -79\% chiamate (da 200/ep a 42/ep)
    \item \textbf{Threshold decay aggressivo}: LLM usage da 90\% (ep 0) a 0\% (ep 100)
    \item \textbf{Model quantization}: Q4\_K\_M quantized → -40\% memory, -40\% latency
    \item \textbf{Async calls}: Non-blocking LLM → overlap con DQN training (-15\% tempo)
\end{itemize}

\textbf{Risultato}: Time per episode ridotto da 42.7s a 28.3s (-33.7\%). GPU memory da 6.8 GB a 5.0 GB (-26.5\%). Total training time da 3.6h a 2.4h.

\subsubsection{Challenge 5: LLM Hallucinations e Action Typos}

\textbf{Problema}: Helper LLM genera azioni inesistenti (8\% typos come \texttt{place\_rock}, 5\% hallucinations come \texttt{collect\_wood}), causando errori e comportamento subottimale.

\textbf{Soluzione}: Sistema di correzione e validazione:
\begin{itemize}
    \item TYPO\_MAP con 13 correzioni comuni (place\_rock → place\_stone)
    \item Fuzzy matching con Levenshtein distance < 2 → auto-correct
    \item Fallback to noop per hallucinations irrecuperabili
    \item Logging hallucination rate per monitoring
\end{itemize}

\textbf{Risultato}: Valid actions aumentate da 87\% a 98\% (+11\%). Error rate complessivo ridotto da 13\% a 2\% (-84.6\%). Hallucination rate medio durante training: 0.02\%.

\subsubsection{Challenge 6: Fine-tuning del Reviewer}

\textbf{Problema}: Necessità di dataset specifico per Crafter con esempi di qualità, bilanciamento tra feedback positivi/negativi, e metriche per valutare utilità.

\textbf{Soluzione}:
\begin{itemize}
    \item Generazione automatica dataset da 50 episodi con Helper
    \item Annotazione semi-automatica basata su euristica (successo/fallimento)
    \item Augmentation dei casi critici (salute bassa, crafting fallito)
    \item Fine-tuning FLAN-T5-base (250M parametri) per 5 epoch
    \item Validation split 80/20 con monitoring BLEU score
\end{itemize}

\textbf{Risultato}: Dataset di ~2,500 esempi, feedback utili nel 68\% dei casi. Validation loss = 0.342. Contributo quantitativo: +6.7\% achievement, +9.1\% reward, -9.5\% convergenza time.

\subsubsection{Sintesi Soluzioni}

Tutte le sfide sono state risolte con successo, come dimostrato dai miglioramenti misurabili:

\begin{table}[h]
\centering
\begin{tabular}{@{}lccc@{}}
\toprule
\textbf{Sfida} & \textbf{Metrica} & \textbf{Prima} & \textbf{Dopo} \\ \midrule
Reward Sparsity & Achievement (0-100 ep) & 0.4 & 1.9 (+375\%) \\
Dataset Quality & Feedback utili & N/A & 68\% \\
Emergency Handling & Death rate & 38\% & 7\% (-81.6\%) \\
LLM Overhead & Time/episode & 42.7s & 28.3s (-33.7\%) \\
Hallucinations & Valid actions & 87\% & 98\% (+11\%) \\
Reviewer Training & Contribution & N/A & +6.7\% achievement \\ \bottomrule
\end{tabular}
\caption{Impatto delle soluzioni implementate}
\end{table}

Queste soluzioni hanno permesso a HeRoN di raggiungere performance superiori (4.8 achievement score) rispetto al baseline (3.2) con significatività statistica (p < 0.01), dimostrando l'efficacia dell'approccio integrato RL-LLM anche in presenza di sfide tecniche complesse.

\subsection{Limitazioni}

Nonostante i risultati positivi, il progetto presenta alcune limitazioni:

\subsubsection{Limitazioni Architetturali}

\begin{enumerate}
    \item \textbf{Pianificazione a breve termine}: Sequenze di 5 azioni limitano la capacità di perseguire obiettivi molto distanti (es. collect\_diamond richiede 50+ azioni coordinate)
    
    \item \textbf{Coverage incompleta}: 6 achievement su 22 (27.3\%) mai sbloccati durante il training, principalmente quelli più avanzati
    
    \item \textbf{Dipendenza da threshold manuale}: Il decay lineare del threshold è una scelta euristica che potrebbe non essere ottimale
    
    \item \textbf{Gestione inventario limitata}: L'Helper non sempre considera vincoli di capacità inventario
\end{enumerate}

\subsubsection{Limitazioni Computazionali}

\begin{enumerate}
    \item \textbf{Overhead LLM}: Tempo di training 75\% superiore rispetto a DQN puro
    
    \item \textbf{Scalabilità}: Con LLM più grandi (es. Qwen2.5-72B o GPT-4) l'overhead diventerebbe proibitivo
    
    \item \textbf{Memoria GPU}: Mantenere DQN + LLM in memoria richiede GPU con ≥8GB VRAM
\end{enumerate}

\subsection{Lavori Futuri}

Il progetto apre diverse direzioni di ricerca futura:

\subsubsection{Miglioramenti Architetturali}

\begin{enumerate}
    \item \textbf{Pianificazione gerarchica}: 
    \begin{itemize}
        \item Helper genera piani ad alto livello (es. "ottieni ferro")
        \item Sub-planner traduce in sequenze di azioni concrete
        \item Permetterebbe achievement complessi come collect\_diamond
    \end{itemize}
    
    \item \textbf{Threshold adattivo}:
    \begin{itemize}
        \item Invece di decay lineare, adattare in base a performance
        \item Aumentare threshold quando DQN fallisce ripetutamente
        \item Ridurre quando DQN è competente
    \end{itemize}
    
    \item \textbf{Memory augmentation}:
    \begin{itemize}
        \item Aggiungere memoria episodica per ricordare strategie di successo
        \item Permettere all'Helper di consultare esperienze passate
    \end{itemize}
    
    \item \textbf{Multi-agent learning}:
    \begin{itemize}
        \item Più NPC che condividono esperienze
        \item Helper centralizzato che apprende da tutti gli agenti
    \end{itemize}
\end{enumerate}

\subsubsection{Ottimizzazioni del Reviewer}

\begin{enumerate}
    \item \textbf{Dataset di qualità superiore}:
    \begin{itemize}
        \item Annotazione manuale da esperti umani
        \item Utilizzo di LLM più grandi per generare feedback di riferimento
        \item Active learning per selezionare esempi informativi
    \end{itemize}
    
    \item \textbf{Reinforcement Learning per Reviewer}:
    \begin{itemize}
        \item Invece di supervised fine-tuning, usare RLHF
        \item Reward basato su miglioramento effettivo dopo feedback
        \item Potrebbe migliorare qualità feedback oltre il 68\% attuale
    \end{itemize}
    
    \item \textbf{Reviewer specializzati}:
    \begin{itemize}
        \item Reviewer diversi per survival, combat, crafting
        \item Ensemble di Reviewer per robustezza
    \end{itemize}
\end{enumerate}

\subsubsection{Estensioni dell'Applicazione}

\begin{enumerate}
    \item \textbf{Altri environment}:
    \begin{itemize}
        \item NetHack: Roguelike complesso
        \item Minecraft: Versione completa
        \item Starcraft II: RTS strategico
        \item Validare generalizzazione dell'approccio
    \end{itemize}
    
    \item \textbf{Multi-task learning}:
    \begin{itemize}
        \item Training simultaneo su più environment
        \item Helper generale che si adatta a task diversi
    \end{itemize}
    
    \item \textbf{Zero-shot transfer}:
    \begin{itemize}
        \item Training su Crafter, test su environment simili
        \item Valutare capacità di trasferimento della conoscenza
    \end{itemize}
\end{enumerate}

\subsubsection{Analisi Teoriche}

\begin{enumerate}
    \item \textbf{Convergenza formale}:
    \begin{itemize}
        \item Dimostrare matematicamente convergenza di HeRoN
        \item Analizzare impatto dell'intervento LLM sulla policy ottimale
    \end{itemize}
    
    \item \textbf{Sample efficiency}:
    \begin{itemize}
        \item Quantificare riduzione sample complexity con LLM
        \item Confrontare con human demonstrations
    \end{itemize}
    
    \item \textbf{Interpretabilità}:
    \begin{itemize}
        \item Analisi qualitativa delle strategie apprese
        \item Visualizzazione delle decisioni Helper vs DQN
        \item Understanding del processo di raffinamento Reviewer
    \end{itemize}
\end{enumerate}

\subsubsection{Applicazioni Pratiche}

L'architettura HeRoN potrebbe essere applicata a:

\begin{itemize}
    \item \textbf{Game AI}: NPC più intelligenti e adattabili nei videogiochi
    \item \textbf{Robotica}: Combinare planning LLM con control RL per task complessi
    \item \textbf{Assistenti virtuali}: Agenti che combinano ragionamento e apprendimento
    \item \textbf{Automazione industriale}: Sistemi che si adattano a nuove situazioni
\end{itemize}

\subsection{Considerazioni Finali}

Questo progetto ha dimostrato con successo che l'architettura HeRoN può essere estesa oltre il suo dominio originale (JRPG a turni) a environment più complessi come Crafter. L'integrazione tra Reinforcement Learning e Large Language Model offre vantaggi significativi in termini di:

\begin{itemize}
    \item Velocità di apprendimento (convergenza 41.5\% più rapida)
    \item Performance finale (+50\% achievement score)
    \item Capacità di pianificazione strategica
    \item Adattabilità a nuove situazioni
\end{itemize}

Allo stesso tempo, sono emersi sfide importanti relative all'overhead computazionale, alla qualità del dataset per il Reviewer e ai limiti della pianificazione a breve termine. Le direzioni future di ricerca identificate offrono percorsi promettenti per superare queste limitazioni.

L'approccio HeRoN rappresenta un passo significativo verso agenti intelligenti che combinano la robustezza dell'apprendimento per rinforzo con la flessibilità e conoscenza generale dei Large Language Model. Man mano che i modelli linguistici diventano più efficienti e capaci, ci aspettiamo che architetture ibride come HeRoN giochino un ruolo sempre più importante nell'IA per giochi, robotica e automazione.

\vspace{1cm}

\noindent\textit{Il codice sorgente, i modelli addestrati e i risultati sperimentali completi sono disponibili nel repository del progetto per consentire la replicabilità e ulteriori sviluppi da parte della comunità di ricerca.}


%% ----------------------------------------------------------------------------
%% BIBLIOGRAFIA
%% ----------------------------------------------------------------------------
%\bibliographystyle{elsarticle-harv}
%\bibliography{bibliography}

\end{document}