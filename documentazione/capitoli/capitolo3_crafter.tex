\chapter{Environment Crafter}

\section{Introduzione a Crafter}

Crafter è un environment di ricerca per il Reinforcement Learning sviluppato come versione semplificata e controllata di Minecraft \cite{hafner2021crafter}. L'environment è stato specificamente progettato per valutare l'intero spettro delle capacità degli agenti RL, dalla sopravvivenza base alla progressione tecnologica complessa.

\subsection{Caratteristiche Principali}

\begin{itemize}
    \item \textbf{Open-world 2D}: Mondo procedurale generato casualmente con terreni variegati
    \item \textbf{Survival game}: Focus su raccolta risorse, crafting e sopravvivenza
    \item \textbf{Osservazioni visive}: Frame RGB di dimensione $64 \times 64 \times 3$
    \item \textbf{22 Achievement}: Obiettivi progressivi che testano diverse abilità
    \item \textbf{Episodi limitati}: Ogni episodio dura al massimo 10,000 step
\end{itemize}

\section{Meccaniche di Gioco}

\subsection{Obiettivi di Sopravvivenza}

Il giocatore deve gestire tre statistiche vitali:

\begin{itemize}
    \item \textbf{Salute (Health)}: Diminuisce quando si viene attaccati da mostri. Se raggiunge zero, l'episodio termina.
    \item \textbf{Cibo (Food)}: Diminuisce costantemente nel tempo. A zero, la salute inizia a diminuire.
    \item \textbf{Acqua (Water)}: Diminuisce costantemente nel tempo. A zero, la salute inizia a diminuire.
\end{itemize}

Per sopravvivere, il giocatore deve:
\begin{enumerate}
    \item Raccogliere cibo (piante, animali)
    \item Bere acqua (esplorando il mondo)
    \item Dormire per rigenerare la salute
    \item Evitare o combattere i mostri
\end{enumerate}

\subsection{Sistema di Progressione}

Crafter implementa un sistema di progressione tecnologica simile a Minecraft:

\begin{enumerate}
    \item \textbf{Raccolta base}: Legno, pietra
    \item \textbf{Costruzione strumenti}: Tavolo di lavoro, fornace
    \item \textbf{Strumenti di pietra}: Piccone, spada di pietra
    \item \textbf{Strumenti di ferro}: Estrazione ferro, crafting strumenti avanzati
\end{enumerate}

Ogni livello di progressione sblocca nuove possibilità e achievement.

\section{Spazio di Stati}

Mentre Crafter fornisce osservazioni visive RGB, nel nostro progetto utilizziamo una rappresentazione strutturata dello stato composta da 43 dimensioni:

\subsection{Inventario (16 dimensioni)}

Conta degli item posseduti dal giocatore:
\begin{verbatim}
[wood, stone, coal, iron, diamond, sapling, 
 wood_pickaxe, stone_pickaxe, iron_pickaxe,
 wood_sword, stone_sword, iron_sword,
 drink, food, health_potion, arrow]
\end{verbatim}

\subsection{Posizione e Orientamento (2 dimensioni)}

\begin{itemize}
    \item Coordinata X (normalizzata)
    \item Coordinata Y (normalizzata)
\end{itemize}

\subsection{Statistiche Vitali (3 dimensioni)}

\begin{itemize}
    \item Livello di salute (0-9)
    \item Livello di cibo (0-9)
    \item Livello di acqua (0-9)
\end{itemize}

\subsection{Achievement (22 dimensioni)}

Vettore binario che indica quali achievement sono stati sbloccati:
\begin{verbatim}
[collect_wood, collect_stone, collect_coal, collect_iron,
 collect_diamond, place_table, place_furnace, place_plant,
 place_stone, defeat_zombie, defeat_skeleton, eat_cow,
 eat_plant, drink_water, make_wood_pickaxe, 
 make_stone_pickaxe, make_iron_pickaxe, make_wood_sword,
 make_stone_sword, make_iron_sword, sleep, wake_up]
\end{verbatim}

\subsection{Rappresentazione Compatta}

La rappresentazione a 43 dimensioni offre diversi vantaggi:
\begin{itemize}
    \item \textbf{Efficienza}: Molto più compatta delle osservazioni RGB ($64 \times 64 \times 3 = 12,288$ valori)
    \item \textbf{Interpretabilità}: Ogni dimensione ha un significato semantico chiaro
    \item \textbf{Facilitazione apprendimento}: Riduce la complessità per l'agente RL
    \item \textbf{Compatibilità LLM}: Facilita la descrizione testuale dello stato per gli LLM
\end{itemize}

\section{Spazio delle Azioni}

Crafter definisce 17 azioni discrete che il giocatore può eseguire:

\subsection{Movimento (4 azioni)}
\begin{itemize}
    \item \texttt{move\_left}: Sposta il giocatore a sinistra
    \item \texttt{move\_right}: Sposta il giocatore a destra
    \item \texttt{move\_up}: Sposta il giocatore verso l'alto
    \item \texttt{move\_down}: Sposta il giocatore verso il basso
\end{itemize}

\subsection{Interazione (2 azioni)}
\begin{itemize}
    \item \texttt{do}: Azione contestuale (raccogliere, attaccare, crafting)
    \item \texttt{sleep}: Dormi per rigenerare salute (richiede essere su erba di notte)
\end{itemize}

\subsection{Posizionamento (4 azioni)}
\begin{itemize}
    \item \texttt{place\_stone}: Posiziona pietra dall'inventario
    \item \texttt{place\_table}: Posiziona tavolo di lavoro
    \item \texttt{place\_furnace}: Posiziona fornace
    \item \texttt{place\_plant}: Pianta un albero
\end{itemize}

\subsection{Crafting (6 azioni)}
\begin{itemize}
    \item \texttt{make\_wood\_pickaxe}: Crea piccone di legno (richiede tavolo)
    \item \texttt{make\_stone\_pickaxe}: Crea piccone di pietra (richiede tavolo)
    \item \texttt{make\_iron\_pickaxe}: Crea piccone di ferro (richiede tavolo + fornace)
    \item \texttt{make\_wood\_sword}: Crea spada di legno (richiede tavolo)
    \item \texttt{make\_stone\_sword}: Crea spada di pietra (richiede tavolo)
    \item \texttt{make\_iron\_sword}: Crea spada di ferro (richiede tavolo + fornace)
\end{itemize}

\subsection{Nessuna Azione (1 azione)}
\begin{itemize}
    \item \texttt{noop}: Non fare nulla (utile per aspettare o concludere sequenze)
\end{itemize}

\section{Sistema di Reward}

\subsection{Reward Nativo (Sparse)}

Crafter fornisce un reward sparso basato sugli achievement:
$$r_{native} = \begin{cases} 
+1 & \text{se achievement sbloccato} \\
0 & \text{altrimenti}
\end{cases}$$

Questo reward è estremamente sparso: in un episodio tipico, il giocatore può sbloccare 0-5 achievement su 22 possibili.

\subsection{Reward Shaping (Dense)}

Per facilitare l'apprendimento, abbiamo implementato un sistema di reward shaping che fornisce segnali più frequenti:

\begin{align}
r_{shaped} &= r_{native} + r_{resources} + r_{health} + r_{tier} + r_{tools} \\
r_{resources} &= 0.1 \times \text{\# nuove risorse raccolte} \\
r_{health} &= 0.05 \times \Delta_{health} \quad (\text{se positivo}) \\
r_{tier} &= 0.05 \times \Delta_{tier} \quad (\text{progressione tecnologica}) \\
r_{tools} &= 0.02 \times \text{\# nuovi strumenti crafted}
\end{align}

Il reward shaping mantiene i seguenti principi:
\begin{itemize}
    \item Non altera gli ottimi della policy (bonus solo per progressi effettivi)
    \item Mantiene lo stesso ordine di grandezza del reward nativo
    \item Fornisce feedback più denso durante l'esplorazione iniziale
\end{itemize}

\section{Achievement e Difficoltà}

I 22 achievement di Crafter sono progettati con difficoltà crescente:

\subsection{Achievement Base (Facili)}
\begin{itemize}
    \item \texttt{collect\_wood}: Raccogliere legno
    \item \texttt{eat\_plant}: Mangiare una pianta
    \item \texttt{drink\_water}: Bere acqua
\end{itemize}

\subsection{Achievement Intermedi}
\begin{itemize}
    \item \texttt{place\_table}: Posizionare tavolo di lavoro
    \item \texttt{make\_wood\_pickaxe}: Creare piccone di legno
    \item \texttt{collect\_stone}: Raccogliere pietra
    \item \texttt{defeat\_zombie}: Sconfiggere uno zombie
\end{itemize}

\subsection{Achievement Avanzati (Difficili)}
\begin{itemize}
    \item \texttt{collect\_iron}: Raccogliere ferro (richiede piccone di pietra)
    \item \texttt{make\_iron\_pickaxe}: Creare piccone di ferro
    \item \texttt{collect\_diamond}: Raccogliere diamante (richiede piccone di ferro)
\end{itemize}

\subsection{Dipendenze tra Achievement}

Molti achievement hanno dipendenze implicite:

\begin{verbatim}
collect_wood -> make_wood_pickaxe -> collect_stone 
    -> make_stone_pickaxe -> collect_iron -> place_furnace
    -> make_iron_pickaxe -> collect_diamond
\end{verbatim}

Questa struttura gerarchica richiede all'agente di apprendere sequenze di azioni complesse e pianificazione a lungo termine.

\section{Sfide per il Reinforcement Learning}

Crafter presenta diverse sfide che lo rendono un benchmark impegnativo:

\begin{enumerate}
    \item \textbf{Sparsità del reward}: Gli achievement sono rari, rendendo difficile l'esplorazione
    
    \item \textbf{Orizzonte lungo}: Alcuni achievement richiedono centinaia o migliaia di azioni
    
    \item \textbf{Dipendenze complesse}: Necessità di completare achievement in ordine specifico
    
    \item \textbf{Gestione risorse}: Bilanciamento tra esplorazione e sopravvivenza
    
    \item \textbf{Variabilità procedurale}: Ogni episodio ha un mondo generato casualmente
    
    \item \textbf{Multi-task}: 22 obiettivi diversi da apprendere simultaneamente
\end{enumerate}

Queste sfide rendono Crafter un banco di prova ideale per valutare architetture avanzate come HeRoN, che combinano RL e ragionamento LLM per affrontare complessità strategiche.
