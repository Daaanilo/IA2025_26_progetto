\chapter{Environment Crafter}

\section{Introduzione a Crafter}

Crafter è un environment di ricerca per Reinforcement Learning, ispirato a Minecraft ma più semplice e controllato. Serve a valutare le capacità degli agenti RL, dalla sopravvivenza base alla progressione tecnologica.

\subsection{Caratteristiche Principali}

\begin{itemize}
    \item \textbf{Open-world 2D}: Mondo generato proceduralmente con terreni vari
    \item \textbf{Survival game}: Raccolta risorse, crafting e sopravvivenza
    \item \textbf{Osservazioni visive}: Frame RGB $64 \times 64 \times 3$
    \item \textbf{22 Achievement}: Obiettivi progressivi che testano varie abilità
    \item \textbf{Episodi limitati}: Durata massima di 10,000 step per episodio
\end{itemize}

\section{Meccaniche di Gioco}

\subsection{Obiettivi di Sopravvivenza}

Il giocatore deve gestire tre statistiche vitali:

\begin{itemize}
    \item \textbf{Salute (Health)}: Diminuisce se attaccato dai mostri, a zero termina l’episodio
    \item \textbf{Cibo (Food)}: Diminuisce col tempo; se a zero, la salute cala
    \item \textbf{Acqua (Water)}: Diminuisce col tempo; se a zero, la salute cala
\end{itemize}

Per sopravvivere, il giocatore deve:
\begin{enumerate}
    \item Raccogliere cibo (piante, animali)
    \item Bere acqua esplorando il mondo
    \item Dormire per rigenerare salute
    \item Evitare o combattere i mostri
\end{enumerate}

\subsection{Sistema di Progressione}

Il sistema di progressione tecnologica include:

\begin{enumerate}
    \item \textbf{Raccolta base}: Legno, pietra
    \item \textbf{Costruzione strumenti}: Tavolo di lavoro, fornace
    \item \textbf{Strumenti di pietra}: Piccone, spada
    \item \textbf{Strumenti di ferro}: Estrazione ferro e crafting avanzato
\end{enumerate}

Ogni livello sblocca nuove azioni e obiettivi.

\section{Spazio di Stati}

Nel progetto usiamo una rappresentazione strutturata a 43 dimensioni per migliorare efficienza, interpretabilità, apprendimento e compatibilità con LLM:

\paragraph{Inventario (16 dimensioni)} Conteggio degli oggetti:
\begin{verbatim}
[wood, stone, coal, iron, diamond, sapling, 
 wood_pickaxe, stone_pickaxe, iron_pickaxe,
 wood_sword, stone_sword, iron_sword,
 drink, food, health_potion, arrow]
\end{verbatim}

\paragraph{Posizione e Orientamento (2 dimensioni)}
\begin{itemize}
    \item Coordinata X (normalizzata)
    \item Coordinata Y (normalizzata)
\end{itemize}

\paragraph{Statistiche Vitali (3 dimensioni)}
\begin{itemize}
    \item Salute (0-9)
    \item Cibo (0-9)
    \item Acqua (0-9)
\end{itemize}

\paragraph{Achievement (22 dimensioni)} Vettore binario degli achievement sbloccati:
\begin{verbatim}
[collect_wood, collect_stone, collect_coal,
 collect_iron, collect_diamond, place_table,
 place_furnace, place_plant, place_stone,
 defeat_zombie, defeat_skeleton, eat_cow,
 eat_plant, drink_water, make_wood_pickaxe,
 make_stone_pickaxe, make_iron_pickaxe,
 make_wood_sword, make_stone_sword,
 make_iron_sword, sleep, wake_up]
\end{verbatim}

\subsection{Spazio delle Azioni}

Crafter prevede 17 azioni discrete:

\paragraph{Movimento (4 azioni)}
\begin{itemize}
    \item \texttt{move\_left}, \texttt{move\_right}, \texttt{move\_up}, \texttt{move\_down}
\end{itemize}

\paragraph{Interazione (2 azioni)}
\begin{itemize}
    \item \texttt{do} (azione contestuale), \texttt{sleep} (rigenera salute, se su erba di notte)
\end{itemize}

\paragraph{Posizionamento (4 azioni)}
\begin{itemize}
    \item \texttt{place\_stone}, \texttt{place\_table}, \texttt{place\_furnace}, \texttt{place\_plant}
\end{itemize}

\paragraph{Crafting (6 azioni)}
\begin{itemize}
    \item \texttt{make\_wood\_pickaxe}, \texttt{make\_stone\_pickaxe}, \texttt{make\_iron\_pickaxe}, 
    \item \texttt{make\_wood\_sword}, \texttt{make\_stone\_sword}, \texttt{make\_iron\_sword}
\end{itemize}

\paragraph{Nessuna Azione (1 azione)}
\begin{itemize}
    \item \texttt{noop} (nessuna azione)
\end{itemize}

\subsection{Sistema di Reward}

\subsubsection{Reward Nativo (Sparse)}

Crafter fornisce un reward sparso basato sugli achievement:
$$
r_{\text{native}} = \begin{cases} 
+1 & \text{se achievement sbloccato} \\
0 & \text{altrimenti}
\end{cases}
$$

Questo reward è estremamente sparso: in un episodio tipico, il giocatore può sbloccare 0-5 achievement su 22 possibili.

\subsubsection{Reward Shaping (Dense)}


Per facilitare l'apprendimento, viene implementato un sistema di reward shaping che fornisce segnali più frequenti:

\begin{align}
r_{shaped} &= r_{native} + r_{resources} + r_{health} + r_{tier} + r_{tools} \\
r_{resources} &= 0.1 \times \text{\# nuove risorse raccolte} \\
r_{health} &= 0.05 \times \Delta_{health} \quad (\text{se positivo}) \\
r_{tier} &= 0.05 \times \Delta_{tier} \quad (\text{progressione tecnologica}) \\
r_{tools} &= 0.02 \times \text{\# nuovi strumenti crafted}
\end{align}

Il reward shaping mantiene i seguenti principi:
\begin{itemize}
    \item Non altera gli ottimi della policy (bonus solo per progressi effettivi)
    \item Mantiene lo stesso ordine di grandezza del reward nativo
    \item Fornisce feedback più denso durante l'esplorazione iniziale
\end{itemize}

\subsubsection{Dipendenze tra Achievement}

Molti achievement hanno dipendenze implicite:

\begin{verbatim}
    collect_wood -> make_wood_pickaxe ->
    collect_stone -> make_stone_pickaxe ->
    collect_iron -> place_furnace ->
    make_iron_pickaxe -> collect_diamond
\end{verbatim}

Questa struttura gerarchica richiede all'agente di apprendere sequenze di azioni complesse e pianificazione a lungo termine.

%\subsection{Sfide per il Reinforcement Learning}
%
%Crafter presenta diverse sfide che lo rendono un benchmark impegnativo:
%
%\begin{enumerate}
%    \item \textbf{Sparsità del reward}: Gli achievement sono rari, rendendo difficile l'esplorazione
%    
%    \item \textbf{Orizzonte lungo}: Alcuni achievement richiedono centinaia o migliaia di azioni
%    
%    \item \textbf{Dipendenze complesse}: Necessità di completare achievement in ordine specifico
%    
%    \item \textbf{Gestione risorse}: Bilanciamento tra esplorazione e sopravvivenza
%    
%    \item \textbf{Variabilità procedurale}: Ogni episodio ha un mondo generato casualmente
%    
%    \item \textbf{Multi-task}: 22 obiettivi diversi da apprendere simultaneamente
%\end{enumerate}
%
%Queste sfide rendono Crafter un banco di prova ideale per valutare architetture avanzate come HeRoN, che combinano RL e ragionamento LLM per affrontare complessità strategiche.
