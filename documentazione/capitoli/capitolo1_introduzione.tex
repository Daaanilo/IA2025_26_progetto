\chapter{Introduzione}

\section{Contesto}

Il presente lavoro rientra nel campo del Reinforcement Learning applicato ai videogiochi, un'area di ricerca in rapida crescita che mira a creare agenti intelligenti capaci di imparare strategie ottimali interagendo con ambienti di gioco.

I videogiochi moderni, soprattutto quelli open-world e di sopravvivenza, presentano sfide complesse che richiedono agli agenti di prendere decisioni strategiche a lungo termine, gestire risorse limitate e adattarsi a situazioni dinamiche. Questi ambienti sono perfetti per testare e validare nuove idee di intelligenza artificiale.

\section{Motivazione e Obiettivi}

L'architettura HeRoN (Helper-Reviewer-NPC) è un approccio innovativo che combina il Reinforcement Learning tradizionale con il ragionamento dei Large Language Model (LLM). Questa architettura è stata inizialmente validata in environment di tipo JRPG (Japanese Role-Playing Game) a turni, dimostrando efficacia nel migliorare le prestazioni degli agenti RL. 

La sfida principale consiste nell'estensione di HeRoN a un contesto molto diverso: il gioco Crafter, un open-world di sopravvivenza che richiede pianificazione a lungo termine, gestione delle risorse e adattamento dinamico. 

\subsection{Obiettivi Primari}

Gli obiettivi principali del lavoro sono:

\begin{itemize}
    \item \textbf{Adattamento architetturale}: Estendere HeRoN dall'environment JRPG a turni a Crafter, un survival game open-world in tempo continuo
    \item \textbf{Fine-tuning del Reviewer}: Adattare il componente Reviewer ai task specifici di Crafter, generando un dataset appropriato e addestrando il modello per feedback efficaci nel survival game
    \item \textbf{Generazione di sequenze}: Modificare l'Helper per generare sequenze di 3-5 azioni coerenti anziché singole decisioni
    \item \textbf{Implementazione DQN}: Sviluppare un agente RL basato su Deep Q-Network ottimizzato per le 17 azioni disponibili e spazio di stati a 43 dimensioni
    \item \textbf{Valutazione comparativa}: Valutare HeRoN quantitativamente rispetto a baseline tradizionali, misurando achievement sbloccati nei 22 obiettivi disponibili
\end{itemize}

\subsection{Obiettivi Secondari}

\begin{itemize}
    \item Determinare il numero ottimale di azioni per sequenza dell'Helper
    \item Analizzare l'impatto del reward shaping sulle prestazioni dell'agente
    \item Implementare meccanismi di re-planning per situazioni critiche (salute bassa, achievement sbloccati)
\end{itemize}

%\subsection{Struttura del Documento}
%
%Il presente documento è organizzato come segue:
%
%\begin{itemize}
%    \item \textbf{Sezione 2 - Architettura HeRoN}: Descrizione dettagliata dell'architettura HeRoN, dei suoi tre componenti (NPC, Helper, Reviewer) e delle loro interazioni.
%    
%    \item \textbf{Sezione 3 - Environment Crafter}: Presentazione dell'environment Crafter, delle sue caratteristiche, dello spazio di stati e azioni, e dei 22 achievement disponibili.
%    
%    \item \textbf{Sezione 4 - Metodologia}: Descrizione della metodologia di implementazione, dalla preparazione dell'environment al training dell'architettura completa.
%    
%    \item \textbf{Sezione 5 - Risultati}: Presentazione e analisi dei risultati sperimentali, con confronti quantitativi tra HeRoN e baseline.
%    
%    \item \textbf{Sezione 6 - Conclusioni}: Sintesi dei risultati ottenuti, discussione delle sfide affrontate e prospettive future.
%\end{itemize}
