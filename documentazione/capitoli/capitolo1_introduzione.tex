\section{Introduzione}

\subsection{Contesto}

Il presente lavoro si inserisce nel contesto del Reinforcement Learning (RL) applicato ai videogiochi, un campo di ricerca in rapida espansione che mira a sviluppare agenti intelligenti capaci di apprendere strategie ottimali attraverso l'interazione con l'ambiente di gioco.

I moderni videogiochi, in particolare quelli di tipo open-world e survival, presentano sfide complesse che richiedono agli agenti di prendere decisioni strategiche a lungo termine, gestire risorse limitate e adattarsi a situazioni dinamiche. Questi ambienti rappresentano un banco di prova ideale per testare e validare nuove architetture di intelligenza artificiale.

\subsection{Motivazione}

L'architettura HeRoN (Helper-Reviewer-NPC) rappresenta un approccio innovativo che combina il Reinforcement Learning tradizionale con le capacità di ragionamento dei Large Language Models (LLM). Questa architettura è stata inizialmente validata in environment di tipo JRPG (Japanese Role-Playing Game) a turni, dimostrando la sua efficacia nel migliorare le prestazioni degli agenti RL attraverso suggerimenti strategici forniti da modelli linguistici.

La sfida principale di questo progetto consiste nell'estendere e validare l'architettura HeRoN in un contesto significativamente diverso: l'environment Crafter, un open-world survival game che richiede capacità di pianificazione a lungo termine, gestione delle risorse e adattamento dinamico.

\subsection{Obiettivi del Progetto}

Il progetto si propone di raggiungere i seguenti obiettivi principali:

\subsubsection{Obiettivi Primari}

\begin{itemize}
    \item \textbf{Adattamento dell'architettura HeRoN}: Estendere l'architettura HeRoN dall'environment JRPG a turni all'environment Crafter, un survival game open-world in tempo continuo.
    
    \item \textbf{Fine-tuning del Reviewer}: Adattare il componente Reviewer ai nuovi task specifici di Crafter, generando un dataset appropriato e addestrando il modello per fornire feedback efficaci nel contesto del survival game.
    
    \item \textbf{Modifica del Helper}: Modificare il comportamento del componente Helper affinché generi sequenze di azioni coerenti (3-5 azioni) anziché singole decisioni, permettendo una pianificazione più strategica.
    
    \item \textbf{Implementazione dell'NPC}: Sviluppare un agente di Reinforcement Learning basato sull'algoritmo Deep Q-Network (DQN) ottimizzato per le 17 azioni disponibili in Crafter e il suo spazio di stati a 43 dimensioni.
    
    \item \textbf{Valutazione delle prestazioni}: Valutare quantitativamente le prestazioni dell'architettura HeRoN completa rispetto a baseline tradizionali, misurando il numero di achievement sbloccati nei 22 obiettivi disponibili in Crafter.
\end{itemize}

\subsubsection{Obiettivi Secondari}

\begin{itemize}
    \item Analizzare il numero ottimale di azioni da suggerire per ciascuna chiamata del Helper.
    
    \item Studiare l'impatto del reward shaping sulle prestazioni dell'agente.
    
    \item Implementare meccanismi di re-planning intelligenti che interrompano le sequenze di azioni in situazioni critiche (salute bassa, achievement sbloccati).
    
    \item Valutare l'efficacia della soglia dinamica che regola la frequenza di consultazione dei componenti LLM durante il training.
\end{itemize}

%\subsection{Struttura del Documento}
%
%Il presente documento è organizzato come segue:
%
%\begin{itemize}
%    \item \textbf{Sezione 2 - Architettura HeRoN}: Descrizione dettagliata dell'architettura HeRoN, dei suoi tre componenti (NPC, Helper, Reviewer) e delle loro interazioni.
%    
%    \item \textbf{Sezione 3 - Environment Crafter}: Presentazione dell'environment Crafter, delle sue caratteristiche, dello spazio di stati e azioni, e dei 22 achievement disponibili.
%    
%    \item \textbf{Sezione 4 - Metodologia}: Descrizione della metodologia di implementazione, dalla preparazione dell'environment al training dell'architettura completa.
%    
%    \item \textbf{Sezione 5 - Risultati}: Presentazione e analisi dei risultati sperimentali, con confronti quantitativi tra HeRoN e baseline.
%    
%    \item \textbf{Sezione 6 - Conclusioni}: Sintesi dei risultati ottenuti, discussione delle sfide affrontate e prospettive future.
%\end{itemize}
